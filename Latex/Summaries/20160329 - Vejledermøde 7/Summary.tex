\input{../Preamble}
\newcommand{\HRule}{\rule{\linewidth}{0.1mm}}

\begin{document}
	\begin{center}
		{\huge \bfseries \textsc{Møde referat nr. 07}}\\
		\textsc{\large 4. semesterprojekt - Gruppe 1}\\[0.3cm]
	\end{center}
	\begin{tabular}{ll}
	\large \textbf{Dato:} & 29/03/2016  \\ % Sted
	\large \textbf{Tid:}  & 14.15-15.15 \\ % Tid
	\large \textbf{Sted:} & Kahn bygningen	\\ % Lokation
	\large \textbf{Deltagere:} & Jens (JN), Jonas (JH), Jonathan (JS), Laimonas (LB), \\
	\large \textbf & Thomas (TN),  Thomas (TS) \& Carl Jakobsen (JA)\\
	\large \textbf{Udeblevet:} & Navn her (Syg) \& Navn her (Ingen udmelding)	\\
	\end{tabular}\\
	\phantom{\,}\hspace{0.1em} \large \textbf{Referat:}
	\begin{enumerate}
		\itemsep 0.3em 
		\item Mødeleder\\
			Jens-Peter
		\item Referent\\
			Laimonas
		\item Godkendelse af sidste referat\\
			Der blev ikke skrevet referat sidste gang grundet et atypisk møde. 

		\item Opfølgning på aktionspunkter
		\begin{itemize}
			\itemsep 0.3em 
			\item Ingen
		\end{itemize}
		
		\item Generelle ting
		\begin{itemize}
			\itemsep 0.3em 
			\item Møde og mangler fra M-ingeniørerne\\
			Snak om møde med M-ingeniørerne, og siden Anders har været heroppe, og da David ikke mener der kommer andre M-ingeniører end ham, vil mødet mødes til en tirsdag, i stedet for at holde et stort møde torsdag 31. marts. 
			Så mødet er flyttet til en tirsdag næste uge, 5. april kl. 14:15. 
			
			Vi mangler en momentkurve for banen i London fra M-ingeniørerne.
			
			\item Overnatning i London\\
			Vi skal måske have taget kontakt om snakke om evt. overnatning på hotel, hvor der ligger flere ved siden af hvor vi skal være. Hvis det er, så er det med at booke med det samme, der var mulighed for et værelse for 2 voksne og 2 ”børn” for 1200kr pr overnatning.
			
			\item Eksamen og London afgang?\\
			Omtalte DSA eksamen som ligger fra d. 28-30, så vi skal have fundet ud af hvornår vi skal være i London, og hvornår eksamenerne skal foreligge. Så dette kan løses. 
			
			Eksamens forløb kommer til at være fritstillet, hvis vi f.eks. vil have en fælles præsentation kan man gøre det, eller hvis ikke kan man også være fri. Dog skal fremlæggelsen være noget som ikke står i rapporten, fordi censor 
			
			\item BMS (Battery Management System)\\
			Vægten ligger anderledes i projektet, og det skal formuleres i opgaveformuleringen og kravspecifikationen.
			Man skal spørge sig selv om man kan få 12, og dertil kan man se i læringsmålene og se om man opfylder disse. Se i bund og grund, skal det være, at man kan få 12 i alle områder. 
			Næste gang diskuteres læringsmålene, så det hele bliver skruet sammen på en rigtig måde, så f.eks. tvivl om BMS dokumentation og lign. bliver løst.
			
			\item Kravspec, dokumentationen, EMC-rapport\\
			-Det vil være en ide at udarbejde 3 dokumentationer, en for rullefeltet og en for bilen og en for GUI.  
			
			-Kravspec og opgavebeskrivelsen vil lægge grund for hele rapporten/dokumentationen, og der skal også indgå ting som f.eks. at vi skriver dokumentationen på engelsk og f.eks. at print skal udarbejdes, loddes, designes og være i færdig tilstand og kunne implementeres i bilen. Disse ting skal med, fordi som sagt, så er projektet anderledes en et normalt 4. semester projekt, og derfor skal det formuleres anderledes. 
			
			-EMC rapport – Hvordan skal den udarbejdes, da print og lign. allerede er udviklet, og da projektet skiller sig ud fra de normale 
			
			
		\end{itemize}			
			
		\item Tidsplan\\
			Der blev afholdt scrum møde efterfølgende, med planlægning af et 2 ugers sprint frem til d. 12/4. Hvor vi efter endt sprint skal kunne være i stand til at sige om vi vil stille op til frankrig. 

		\item Tidspunkt for næste møde\\
			12/04/2016 - 14.15-15.15
		\item Evt.
	\end{enumerate}
\end{document}