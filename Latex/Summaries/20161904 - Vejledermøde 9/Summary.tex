\input{../Preamble}
\newcommand{\HRule}{\rule{\linewidth}{0.1mm}}

\begin{document}
	\begin{center}
		{\huge \bfseries \textsc{Møde referat nr. 09}}\\
		\textsc{\large 4. semesterprojekt - Gruppe 1}\\[0.3cm]
	\end{center}
	\begin{tabular}{ll}
	\large \textbf{Dato:} & 19/04/2016  \\ % Sted
	\large \textbf{Tid:}  & 14.15-15.30 \\ % Tid
	\large \textbf{Sted:} & Kahn bygningen	\\ % Lokation
	\large \textbf{Deltagere:} & Jens (JN), Jonas (JH), Jonathan (JS), Laimonas (LB), \\
	\large \textbf & Thomas (TN),  Thomas (TS) \& Carl Jakobsen (JA)\\
	\large \textbf{Udeblevet:} & Navn her (Syg) \& Navn her (Ingen udmelding)	\\
	\end{tabular}\\
	\phantom{\,}\hspace{0.1em} \large \textbf{Referat:}
	\begin{enumerate}
		\itemsep 0.3em 
		\item Mødeleder\\
			Jens-Peter
		\item Referent\\
			Laimonas
		\item Godkendelse af sidste referat\\
			Møde referat 09 gennemgås ved næste møde: 26 april. 

		\item Opfølgning på aktionspunkter
		\begin{itemize}
			\itemsep 0.3em 
			\item Ingen
		\end{itemize}
		
		\item Generelle ting
			\begin{itemize}
				\item Læringsmål
				Carl vil lave en opgaveformulering, som har de rette ting - Dertil vil kravene laves derudfra.
				Censor får normalt kun læringsmål, dokumentation og raport...
			
				\item E4PRJ4
				
				\begin{itemize}
				\item Optimering og test
				Enhederne skal være bedre testet end de normalt er...
				De skal kunne fungere i de 4 dage, hvor man kører...
				\item Eco marathon dokumentation skal også udføres
				\item Pit-stop udseende, materialer der skal bruges, back-up etc.
				\item Måske skal E-erne tage hånd om M-erne, så man ved hvem der f.eks. har committet sig til en opgave, hvornår de bliver færdige og lign. Også kan man holde styr på / holde hånd op om deadlines...
					\begin{itemize}
						\item Carl har tilbudt at "holde styr" på M-erne.
					\end{itemize}
				
				\item Hvordan skal dokumentationen udføres for BMS, og hvilke test skal til for at vide, at det er færdigt. 
				\item Samme gælder for andre enheder... Hvornår er undersøgelsen færdig, hvornår kan man sige, at de er klar...
					-Laves til krav, accept etc.
					
				\item Huske at dokumentere, at der er brugt tid på at lave reveiw.
				\item Normal også omtalt i rapporten, f.eks. scrum og bl.a. reveiw.
				
				\item Carl vil læse dokumentationen og markere alt det der falder i øjnene. 
				
				\item Hvis der foreligger en kontakt person til M-ingeniører ville det være nemmest. 
				
				\item Motorcontroller printet skal helst være hjemme hurtigst muligt, så måske vil Carl tage en snak med Torben og få en hurtig levering.
					-Carl har sagt, at vi kan få printet inden for 1 uge. Dette kan omtales med Torben for hvordan det passer bedst.
				
				\item BMS => Der kan refereres til den gamle systemarkitektur
				
				\item BMS skal ses som en blackbox... Dvs. komponenter / essentielle ting skal blot omtales og argumenteres for.
				
				\item Bare være sikre på at der er en som går og laver noget omkring 
				
			\end{itemize}
			\item I4PRJ4
			Tage en snak om læringsmålene med Paul, og have snakket det igennem med ham, fordi det er lidt specielt. 
		\end{itemize}
			
		\item Tidsplan\\
			Sprint 5 er igangværende. 

		\item Tidspunkt for næste møde\\
			26/04/2016 - 14.15-15.15
	\end{enumerate}
\end{document}