\input{../Preamble}
\newcommand{\HRule}{\rule{\linewidth}{0.1mm}}

\begin{document}
	\begin{center}
		{\huge \bfseries \textsc{Møde referat nr. 2}}\\
		\textsc{\large 4. semesterprojekt - Gruppe 1}\\[0.3cm]
	\end{center}
	\begin{tabular}{ll}
	\large \textbf{Dato:} & 09/02/2016  \\ % Sted
	\large \textbf{Tid:}  & 14:15-16:00 \\ % Tid
	\large \textbf{Sted:} & Edison		\\ % Lokation
	\large \textbf{Deltagere:} & Jens (JN), Jonas (JH), Jonathan (JS), Laimonas (LB), \\
	\large \textbf & Thomas (TN),  Thomas (TS) \& Carl Jakobsen (JA)\\
	\large \textbf{Udeblevet:} & Ingen
	\end{tabular}\\
	\phantom{\,}\hspace{0.1em} \large \textbf{Referat:}
	\begin{enumerate}
		\itemsep 0.3em 
		\item Mødeleder\\
			Jens-Peter
		\item Referent\\
			Laimonas
		\item Godkendelse af sidste referat\\
			Ingen indsigelser fra gruppens deltagere - Referatet er godkendt. 
			
		\item Generelle ting\\
		\begin{itemize}
			\itemsep 0.3em 
			\item Valg af microcontroller\\
			Der er enighed om at benytte "CY8CKIT-059 PSoC® 5LP" som MCU, som skal bestilles.
			
			\item Bestilling af ting\\
			Man udfylder et elektronisk papir som fås fra værkstedet(Laimonas finder det, og ligger det op på GitHub), dette opfyldes med informationer, om det som ønskes købt. Derefter sendes det til Carl, hvor han elektronisk skriver under på bestillingen. Så medbringes dette til værkstedet og man kan bestille ting hjem.
			
			\item Bilen samt rullefelt\\
			-Vi påbegynder bare at skilde bilen af, når den fås hjem, hvilket højst sandsynligt er i morgen(Onsdag, 10-02-16).
				
			-Rullefeltet er nemmestet at teste når motoren er i bilen (den løftes blot op), så fås også virkningsgraden af det hele (rullemodstand, gearing m.m).
			
			\item Til selve SEM løbet\\
			Man skal mindst have 2 velfungerende set af alle komponenter / systemer, 1 man tager med til løbet, og et som man kan benytte til at teste systemet. 

			\item Batterier\\
			Vi bestiller bare nogle batterirer, og hvis der forekommer ændringer bestiller vi nye batteri til april.
			Batteristyringen antages at fungere, men der er en som skal kende til hele styringen, og dertil kunne fikse problemer når man befinder sig i London. 
			
			\item Kravspecifikation\\
			Kravspecifikation tager vi blot udgangspunkt i de foreliggende regler fra Shell. Dertil kan man så vælge at ydeligere specificere kravene for diverse dele af systemet om nødvendigt. F.eks. skal rullefeltet også specificeres. 
			
			\item SCRUM / ZENHUB\\
			Der er efter vejledermødet (slut 14:45), bl.a. diskuteret issues der kunne skulle medtages i sprintet, dertil blev disse issues estimeret med arbejds timer og tildelt personer.
		\end{itemize}
			
		\item Opfølgning på aktionspunkter
		\begin{itemize}
			\itemsep 0.3em 
			\item Lav en nogenlunde arbejdsfordeling\\
				Der er blevet tilføjet issues og fordelt arbejde i ZENHUB.
			\item Tjek ZEN-hub for opgaver\\
				Der er foretageten oprydning af ZENHUB boardet, tilføjet tænkte issues og dertil påbegyndt sprint 2.
			\item Indsæt aktionspunkt(tjek mødeindkaldelse)\\
				Tekst
		\end{itemize}
		\item Update fra grupperne\\
			Der er ingen grupper endnu.
		\item Tidsplan\\
			Der er ikke udformet en tidsplan for deadlines endnu. 
			
		\item Nye aktionspunkter til næste møde\\
		\begin{itemize}
			\itemsep 0.3em 
		\item Eksamens tilmelding for Jonas\\
		Hvad der skal gøres med Jonas(IKT) angående blackboard semester projekt tilmelding? Alle E'erne forbeholder sig i en gruppe på blackboard(E4PRJ4), Carl er også med, men Jonas er ikke - Sidste år var det er krav at man havde tilmeldt sig blackboard gruppe for at kunne være med til projekt eksamen.
		
		\item Kontakt til omverdenen\\
		-Sende en mail til Shell om spørgsmål og lignende bl.a. angående batterier (må der forefindes 2 batterier, eller 1 batteri til battery electric class).  
		-Der skal tages kontakt til maskiningeniørerne og finde ud af hvad der gøres med horn - Er det os der står for det, eller dem?
		\end{itemize}
		
		\item Tidspunkt for næste møde\\
			16/02/2016 - 14:15-15:15.
		\item Evt.
	\end{enumerate}
\end{document}