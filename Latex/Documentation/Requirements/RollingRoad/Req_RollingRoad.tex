\section{Rolling Road}
This section gives an overview over the requirements for the Rolling Road. The description Rolling Road covers both the mechanically build system and the implemented hardware. 

\subsection{Functional requirements}
Table \ref{Functional:Rolling Road} will provide the functional requirements for the Rolling Road.

\begin{table}[h!]
	\centering
	\begin{tabular}{|p{2 cm}|p{4 cm}|p{3 cm}|}
	\hline
	\textbf{(Req. \#)} & \textbf{Description} & \textbf{Comments} \\\hline
	RR F1	& Must be able to measure torque, ampere and voltage for both the engine and the generator  &   \\\hline
	RR F2	& Must be able to regulate the torque of the generator &   \\\hline
	RR F3	& Must be able to display the results collected on a GUI & For more info, see \ref{sec:rollingroadgui}  \\\hline
	RR F4	& Must be able to communicate with a PC &   \\\hline
	RR F5	& Should be able to measure the velocity of the wheel &   \\\hline
	\end{tabular}
	\label{Functional:Rolling Road}
	\caption{Functional requirements for Rolling Road}
\end{table}

\subsection{Non-functional requirements}
Here, the presented information is the non-functional requirements for the Rolling Road.

\begin{table}[h!]
	\centering
	\begin{tabular}{|p{2 cm}|p{4 cm}|p{3 cm}|}
		\hline
		\textbf{(Req. \#)} & \textbf{Description} & \textbf{Comments} \\\hline
		RR nF1	& Must be able to measure the current in the Load System up to a maximum of 21 A  &   \\\hline
		RR nF2	& The generator must have a maximum power dissipation of 200w &   \\\hline
		RR nF3	& Must be able to measure af maximum of 6000 RPM &   \\\hline
		RR nF4	& Must be able to perform an automatic shutdown when the current limit is reached &   \\\hline
		RR nF5	& The regulation must have a bandwidth of xxx\fxnote(Husk at ændre) &   \\\hline
	\end{tabular}
	\label{Nonfunctional:RollingRoad}
	\caption{Non-functional requirements for Rolling Road}
\end{table}