\section{Protocol}

Each command is termintad by a newline character, and each value in a command is separated by a whitespace.

Overview of packets:
\begin{table}[h!]
	\centering
	\label{Protocol:overview}
	\begin{tabular}{l|llll}
		\# & Description 		& Command    		& Direction             & Example     		\\\hline
		0  & Handshake   		& RollingRoad 		& PC $\rightarrow$ PSoC & 0 RollingRoad 	\\
		1  & Unit description 	& <int> <str> <str> & PSoC $\rightarrow$ PC & 1 0 Time Seconds 	\\
		2  & Stop        		&            		& PC $\rightarrow$ PSoC	& 2        			\\
		3  & Information 		& <double> ... <double>	& PSoC $\rightarrow$ PC & 3 2 3 	    	\\
		4  & Torque control 	& <double>    			& PC $\rightarrow$ PSoC & 4 2.5  				\\
	\end{tabular}
	\caption{Overview for commands sent between PSoC og PC}
\end{table}

The table \ref{Protocol:overview} will give an on overview regardring the protocol that will be used between PC and PSoC. The chosen protocol will always start by sending an integer that corresponds with the given number from the above table.  The next thing that will be send is the column "Commando" that corresponds with the given number. The direction column shows the direction of the signals, where from and where to the signal will flow.

\subsection{Handshake}
The handshake command is used to establish that the connection between the PC and the embedded platform is correct. This is to ensure that it will not register an arbitrary unit and for that unit to send values to the PC.  

\subsection{Unit description}
The unit description command will give the PC, with the 3 kinds of information. The first character in the transmitted message will be the command number, here 1. The first real information sent here, relates to the information command. The integer here sent, will define the place of the information in the "array" that is sent. The next thing that will send is the measurement the PC can expect, fx. time, torque etc. The final thing that will be send is the unit for this particular information. This is implementet so that the information message is as short as possible.  

\subsection{Stop}
The stop command is used for stopping the data transmission. The stop command is called either by pressing a button on GUI or closing the program on the PC.
When stop command is issued and data transmission is once again wanted, the protocol will start from the top, by sending the handshake command again. 

\subsection{Information}
The information command will be the command where the all the information will be send. Thismeaning that this command is going to be send very often and therefore this message should be as short as possible. This is done by sending only the information needed, which is just the values of the information. To ensure that the PC and the PSoC communicates correctly here, the unit description command is used. 

\subsection{Torque control}
The torque control command  can change the input to the regulator. This is to change how much the generator is loaded and thereby how much the engine is strained. This parameter can be changed on the GUI. 
          

In figure \ref{fig:TimingDiagram}, the flow of communication can be seen.
Starting with the handshake initiated by the PC, replying with unit descriptions for this transfer. When the connection is established after the handshake, the PSoC will send the information with a frequency of xx\fxnote(Fix) Hz. On the GUI, the user can choose to regulate the torque, which then will be send and understood on the PSoC. Furthermore, the user have the ability to stop the data transfer, by pressing the stop button or by closing the program. When the command Stop is send, the PSoC will cease the data transfer.

\begin{figure}
\centering
\includegraphics[width=0.5\linewidth]{Protocol/TimingDiagram}
\caption{}
\label{fig:TimingDiagram}
\end{figure}

