\newpage
\section{Power Sensor}
The purpose of this block is to measure the electrical power delivered to the car's propulsion system. The power sensor must be able to measure both the delivered voltage V and  current I, as the power P is given by:
\begin{equation}
	P = V \cdot I
\end{equation}

\subsection{Design}
The design is split in two due to the fact that the sensor must measure two parameters.

\textbf{Current Transducer}\\
The current is measured using a Hall-effect based current transducer of the type LTS 15-NP. The internal circuit of the component is seen on Figure \ref{fig:LTS_internal_circuit} below. The transucer measures the strength of the magnetic field induced by the input current and converts it to a voltage. This voltage is amplified and given as the transducer's output. The internal amplifier must be supplied with 5 VDC in order for the transducer to measure properly.

\begin{figure}[H]
	\centering
	\includegraphics[width=0.5\linewidth]{Hardware/Pictures/LTS_circuit}
	\caption{LTS 15-NP Internal circuit}
	\label{fig:LTS_internal_circuit}
\end{figure}

The LTS 15-NP is gives an output-voltage which is proportional to the input-current and lies in the range between $\SI{0.5}{\volt}$ and $\SI{4.5}{\volt}$. The transducer contains 6 pins which can be connected in different ways in order to change the number of primary turns in the coil - and thus changing the primary nominal current rms I\textsubscript{PN}. This means that the input's measuring range can be configured as the output-voltage V\textsubscript{out} can be described using the following formula:
\begin{equation}
	V_{out} = 2.5 \pm \left( 0.625 \cdot \frac{I_{P}}{I_{PN}} \right)
	\label{eq:current_transducer}
\end{equation}

Where I\textsubscript{P} is the transducer's input current. The formula above can be used to plot the output on Figure \ref{fig:LTS_output} below. The accuracy of the transducer is specified to 0.2\%.

\begin{figure}[H]
	\centering
	\includegraphics[width=0.4\linewidth]{Hardware/Pictures/LTS_output}
	\caption{LTS 15-NP Output voltage}
	\label{fig:LTS_output}
\end{figure}

The transducer's pins are configured in such a way that the the primary nominal current rms I\textsubscript{PN} is equal to $\pm \SI{15}{\ampere}$. This allows for the largest measuring range possible. Using this number and equation \ref{eq:current_transducer} it is possible to calculate the range wherein the transducer measurements are linear. This configuration is shown below:

\begin{figure}[H]
	\centering
	\includegraphics[width=0.6\linewidth]{Hardware/Pictures/PowerSensor_Current}
	\caption{Method of current-measurement in the Power Sensor}
	\label{fig:PowerSensorCurrent}
\end{figure}

The minimum-current which the transducer can measure correctly before it enters saturation can be calculated as:
\begin{equation}
	0.5 = 2.5 + \left( 0.625 \cdot \frac{I_{Pmini}}{15} \right) \quad \Rightarrow \quad I_{Pmini} = \SI{48}{\ampere}
\end{equation}

The maximum-current which the transducer can measure correctly before it enters saturation can be calculated as:
\begin{equation}
	4.5 = 2.5 + \left( 0.625 \cdot \frac{I_{Pmaxi}}{15} \right) \quad \Rightarrow \quad I_{Pmaxi} = \SI{-48}{\ampere}
\end{equation}

The transducer's measuring-resolution can be calculated as a output-difference per the corrosponding input-difference. The difference is found using the minimum and maximum range values from the calculations above:
\begin{equation}
	\Delta V = \frac{\SI{4.5}{\volt} - \SI{0.5}{\volt}}{\SI{48}{\ampere} - (\SI{-48}{\ampere})} = \SI[per-mode = fraction]{41.67}{\milli \volt \per \ampere}
\end{equation}

\textbf{Voltage Transducer}\\
The voltage delivered to the motor is measured using a voltage divider consisting of two resistors. The voltage didiver must be placed parallel to the motor to correctly measure the voltage.

\begin{figure}[H]
	\centering
	\includegraphics[width=0.5\linewidth]{Hardware/Pictures/PowerSensor_Voltage}
	\caption{Method of voltage-measurement in the Power Sensor}
	\label{fig:PowerSensorCurrent}
\end{figure}

According to the requirements the car's motor operates with a supply-voltage between $\SI{0}{\volt}$ and $\SI{50}{\volt}$. This range must be rescaled to a range between $\SI{0}{\volt}$ and $\SI{5}{\volt}$ as this is the range which can be measured by the PSoC's ADC.

The voltage divider can be designed by choosing either pf the resistor's values. For this calculation, however, the value of R\textsubscript{1} is choosen to be equal to $\SI{10}{\kilo \ohm}$. The desired value of the resistor R\textsubscript{2} can then be calculated using the following formula:
\begin{equation}
	\begin{split}
	V_{in} &= V_{motor} \cdot \frac{R_2}{R_1 + R_2}\\
	\\
	\SI{5}{\volt} &= \SI{50}{\volt} \cdot \frac{R_2}{\SI{10}{\kilo \ohm} + R_2} \quad \Rightarrow \quad R_2 = \SI{1.1}{\kilo \ohm}
	\end{split}
\end{equation}
For simplicity the resistor's value is rounded to $\SI{1}{\kilo \ohm}$ which means that the highest voltage on the PSoC's input can be calculated as:
\begin{equation}
	V_{in} = \SI{50}{\volt} \cdot \frac{ \SI{10}{\kilo \ohm} }{ \SI{1}{\kilo \ohm} + \SI{10}{\kilo \ohm}} = \SI{4.545}{\volt}
\end{equation}

\subsection{Implementation}
The Power Sensor's circuit can be implemented by combining the two designs above. The product is seen on the circuit-diagram below. All four motor-connections can be connected to using banana-plugs. The voltage- and current-measurements are connected to the PSoC through analog low-pass filters.

\begin{figure}[H]
	\centering
	\includegraphics[width=0.7\linewidth]{Hardware/Pictures/PowerSensor_circuit}
	\caption{Implemented circuit design on Rolling Road}
	\label{fig:PowerSensor_circuit}
\end{figure}

The diodes D1 and D2 are of the type 1N4007 and are used to protect the circuit as the node between them becomes limited to a voltage in the range between \SI{-0.7}{\volt} and \SI{5.7}{\volt}. This means that a potential over-current will flow through these diodes instead of damaging subsequent circuits. The inductor L1 is added for EMC-reasons \fxnote{cite til EMC-rapport}.

\subsection{Unity test}
The unity test of the Power Sensor will only be dealing with the current-measurement. The voltage-measurement is mostly handled by the PSoC's ADC and isn't very relevant for this section. The test is performed by connecting the LTS 15-NP in series with a variable resistor and a voltage-source. The resistor can then be used to control the amount current into the transducer. The transducer's output can then be measured using a volt-meter. The complete setup is shown below:

\begin{figure}[H]
	\centering
	\includegraphics[width=0.55\linewidth]{Hardware/Pictures/PowerSensor_test}
	\caption{Test-setup for Power Sensor (current measurement)}
	\label{fig:PowerSensor_test}
\end{figure}

The distance between the measured amount of current must be larger than the transducer's resolution. This results in the fact that the measured currents must be quite large. For the larger currents a variable power resistor is then used to prevent the setup from overheating. This variable power resistor is adjusted to the correct resistance using an ohm-meter. Furthermore, the current is limited by the voltage-source which can deliver up to \SI{3}{\ampere}.

The results from the test are visualised below using Microsoft Excel's plotting tool. The measured values are held up against the theoretical values (values calculated using the formula in the datasheet).

\begin{figure}[H]
	\centering
	\includegraphics[width=0.9\linewidth]{Hardware/Pictures/PowerSensor_graph}
	\caption{LTS 15-NP output versus input}
	\label{fig:PowerSensor_graph}
\end{figure}

In order to let \SI{0}{\ampere} flow through the transducer, the voltage-source is simply disconnected from the setup. By performing linear regression on the measured data an output-equation can be derived. This equation is held against the one given in the datasheet:
\begin{equation}
	\begin{split}
			V_{out,specified} &= 2.5 + \left( 0.625 \cdot \frac{I_{in}}{\SI{15}{\ampere}} \right) = 0.0417 \cdot I_{in} + 2.5\\
			\\
			V_{out,measured} &= 0.0417 \cdot I_{in} + 2.5
	\end{split}
\end{equation}

It is seen that the according to the regression the output is exactly as it is specified in datasheet. However, this may very well be due to the fact that the used volt-meter (Tenma 72-7765) doesn't have a large enough precision. The measured data is seen on the table below:
\begin{table}[H]
	\centering
	\begin{tabular}{|c|c|c|c|}
		\hline
		\textbf{R}  & \textbf{I\textsubscript{in}}    & \textbf{V\textsubscript{out,specified}}     & \textbf{V\textsubscript{out,measured}}    \\ \hline
		& 0    & 2.500 & 2.50 \\ \hline
		40 & 0.5  & 2.521 & 2.52 \\ \hline
		20 & 1.00 & 2.542 & 2.54 \\ \hline
		10 & 2.00 & 2.583 & 2.58 \\ \hline
		8  & 2.50 & 2.604 & 2.60 \\ \hline
		7  & 2.86 & 2.619 & 2.62 \\ \hline
	\end{tabular}
	\caption{Measured data for the Power Sensor's unity test}
\end{table}