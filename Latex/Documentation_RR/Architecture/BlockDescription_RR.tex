\newpage
\section{Block description}
The blocks in the BDD are described as follows:
\begin{itemize}
	\item \textbf{Voltage Adaptor}\\
	This block converts 230 VAC to 12 VDC in order to control the system. This allows the system to be directly powered by the power grid.
	\item \textbf{Roll Stand}\\
	This block contains the various mechanical subsystems.
	\begin{itemize}
		\item \textbf{Roll}\\
		The cylinder which is rotated by the vehicle.
		\item \textbf{Generator}\\
		A DC-generator which generates a voltage proportional to the angular velocity.
		\item \textbf{Torque Sensor}\\
		Measures the mechanical power given from the subject's wheel (angular velocity and torque) and converts the measurements to electrical signals which can be read by the Control Unit.
	\end{itemize}
	\item \textbf{Control Unit}\\
	Measures the inputs from the sensors in the system in order to calculated the subject's performance. The results from the calculations are passed on to the computer using the UART-protocol specified in Section \vref{sec:UART}. Furthermore, it also regulates the Load System using an PID-controller.
	\item \textbf{Control System}\\
	This block contains the various electrical subsystems.
	\begin{itemize}
		\item \textbf{Signal Converter}\\
		Converts the generated signals from the Torque Sensor to signals which can be read correctly by the Control Unit.
		\item \textbf{Level Converter}\\
		Description: Converts the 12 VDC voltage to 5 VDC in order to power the components in the system which requires 5 V to function.
		\item \textbf{Power Sensor}\\
		Measures the electrical power given to the subject's motor and converts the measurements to signals which can be read by the Control Unit.
	\end{itemize}
	\item \textbf{Load System}\\
	An electrical system where the generated voltage from the Generator can be deposited by turning the electrical energy into heat.
\end{itemize}