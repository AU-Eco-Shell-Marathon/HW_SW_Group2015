\section{Functional Requirement Rolling Road}
Table \ref{FREQ_RR} will provide the functional requirements for the Rolling Road.
\begin{table}[h!]
	\label{FREQ_RR}
	\centering
	\begin{tabular}{|p{2 cm}|p{10 cm}|p{2 cm}|}
		\hline
		\textbf{Req. \#} & \textbf{Description} & \textbf{Comments} \\\hline
		RR\_F1
		& The system must be able to hold a constant force on the subject.
		&  \\ \hline
		RR\_F2
		& The system must be able to change the desired force exerted on the subject. 
		&  \\ \hline
		RR\_F3
		& The system must turn off the Load System if the current through it becomes too big (overcurrent specified in RR\_NF3).
		&  \\ \hline
		RR\_F4
		& The system must be able to measure the power of the subject on test.
		&  \\ \hline
		RR\_F5
		& The system must be able to measure force and speed of the subject on test.
		&  \\ \hline
		RR\_F6
		& The system must be able to calculate the efficiency of the subject on test. 
		&  \\ \hline
		RR\_F7
		& The system must be able to adjust Rolling Road's sensor and set them all to zero. 
		&  \\ \hline
		RR\_F8
		& The Load system contain a constant torque.
		&  \\ \hline
		RR\_F9
		& The system must be able to remember settings after shutdown.
		&  \\ \hline
		RR\_F10
		& The system must be able to controlled through Serial communication.  
		&  \\ \hline
	\end{tabular}
	\caption{Functional requirements concerning Rolling Road}
\end{table}
\newpage