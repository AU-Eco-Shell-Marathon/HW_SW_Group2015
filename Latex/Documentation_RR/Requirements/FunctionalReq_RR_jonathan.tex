\section{Functional Requirement Rolling Road}
\fxnote{kan vi gøre noget ved afstanden mellem overskrift og tabel? - TN}
\begin{table}[h!]
	\label{FREQ_AU2}
	\centering
	\begin{tabular}{|p{2 cm}|p{10 cm}|p{2 cm}|}
		\hline
		\textbf{Req. \#} & \textbf{Description} & \textbf{Comments} \\\hline
		RR\_F1
		& The system must be able to hold a constant force on the subject.
		&  \\ \hline
		RR\_F2
		& The system must be able to change the desired force exerted on the subject. 
		&  \\ \hline
		RR\_F3
		& The system must turn off the Load System if the current through it exceeds 16 A.
		&  \\ \hline
		RR\_F4
		& The system must be able to measure the power of the subject on test.
		&  \\ \hline
		RR\_F5
		& The system must be able to measure force and speed of the subject on test, (to find the mechanical force). \fxnote{Det i parentes er vel ligegyldigt - TN}
		&  \\ \hline
		RR\_F6
		& The system must be able to calculate the efficiency of the subject on test. 
		&  \\ \hline
		RR\_F7
		& The system must be able to adjust Rolling Road's sensor and set them all to zero. 
		&  \\ \hline
		RR\_F8
		& The Load system must be controlled with a PID-regulator.
		&  \\ \hline
		RR\_F9
		& The PID-regulator must have anti windup. \fxnote{Er det nødvendigt at have med som krav? - TN}
		&  \\ \hline
		RR\_F10
		& The system must be able to remember settings after shutdown. Settings which Rolling Road must remember: PID-constants, desired force and offset values of all sensors. \fxnote{Omformuler! - TN}
		&  \\ \hline
		RR\_F11
		& The system must be controlled from the Computer through Serial communication.  
		&  \\ \hline
	\end{tabular}
	\caption{Functional requirements concerning Rolling Road}
\end{table}
\newpage