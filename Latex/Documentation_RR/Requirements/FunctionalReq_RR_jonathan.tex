\section{Functional Requirement Rolling Road}

\begin{table}[h!]
	\label{FREQ_AU2}
	\centering
	\begin{tabular}{|p{2 cm}|p{10 cm}|p{2 cm}|}
		\hline
		\textbf{Req. \#} & \textbf{Description} & \textbf{Comments} \\\hline
		RR\_F1
		& Must be able to hold a constant force on the subject.
		&  \\ \hline
		RR\_F2
		& Must be able to change the wanted force on the subject. 
		&  \\ \hline
		RR\_F3
		& Must turn off Load system if the current through it exceeds 16 A.
		&  \\ \hline
		RR\_F4
		& Rolling Road must be able to measure the power of the subject.
		&  \\ \hline
		RR\_F5
		& Rolling Road must be able to measure force and speed of the subject, to find the mechanic force.
		&  \\ \hline
		RR\_F6
		& Rolling Road must be able to calculate the efficiency of the subject. 
		&  \\ \hline
		RR\_F7
		& Rolling Road can adjust Rolling Road's sensor and set them all to zero. 
		&  \\ \hline
		RR\_F8
		& Load system must be controlled with a PID regulator.
		&  \\ \hline
		RR\_F9
		& PID regulator must have anti windup.
		&  \\ \hline
		RR\_F10
		& Must be able to remember settings after shutdown. Things Rolling Road must remember; PID constants, wanted force and offset values of all sensors.
		&  \\ \hline
		RR\_F11
		& Rolling Road can be controlled through Serial communication.  
		&  \\ \hline
	\end{tabular}
	\caption{Functional requirements concerning Rolling Road}
\end{table}
