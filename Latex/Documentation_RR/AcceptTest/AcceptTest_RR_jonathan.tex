\chapter{Accept test}

\begin{table}[h!]
	\centering
	\label{my-label}	
	\begin{tabular}{|p{1.5 cm}|p{2.1 cm}|p{2.1 cm}|p{2.1 cm}|p{2.1 cm}|}
		\hline
		Test of requirements: 
		& \multicolumn{4}{l|}{Connect to Rolling Road. Calibrate all sensor. Test EEPROM.} \\ \hline
		Setup 
		& \multicolumn{4}{l|}{Rolling Road is turn on and connected to a PC with the GUI application.} \\ \hline
		Test NR:
		& \multicolumn{4}{l|}{1} \\ \hline
		\textbf{Steps} & \textbf{action} & \textbf{Expected result} & 
		\textbf{Actual result} & \textbf{Accept/ Comment} \\ \hline
		1 
		& Click “select” and chose the right com port. 
		& Allot of live data, of all the sensor on the Rolling Road
		&
		& \\ \hline
		2
		& Click Calibrate.
		& All sensor value reset to zero.
		&
		& \\ \hline
		3
		& Change PID values and Force.
		& The live data of setforce change. Maybe PIDval also change.
		&
		& \\ \hline
		4
		& Restart Rolling Road, and reconnect (use step 1)
		& Live data is still calibrate.
		&
		& \\ \hline
	\end{tabular}
	\caption{Something}
\end{table}


\begin{table}[h!]
	\centering
	\label{my-label}	
	\begin{tabular}{|p{1.5 cm}|p{2.1 cm}|p{2.1 cm}|p{2.1 cm}|p{2.1 cm}|}
		\hline
		Test of requirements: 
		& \multicolumn{4}{l|}{Test PID regulator. Test sensors.} \\ \hline
		Setup 
		& \multicolumn{4}{l|}{Is connect to a GUI and test subject is on the Rolling Road.} \\ \hline
		Test NR:
		& \multicolumn{4}{l|}{2} \\ \hline
		\textbf{Steps} & \textbf{action} & \textbf{Expected result} & 
		\textbf{Actual result} & \textbf{Accept/ Comment} \\ \hline
		1 
		& Change PID values to 0 2 0. 
		& Nothing of interest.
		&
		& \\ \hline
		2
		& Set force after table XX and regulate the speed on the subject also after table XX (Read the next step before start)
		& The live data of the force should change to the right value.
		&
		& \\ \hline
		2.1
		& Every step in step 2. Note all values and calculate manual effect and efficiency. $ U*I=E_el $;$ V*F=E_mek $; $ Efficiency=(E_el/E_mek)*100 $
		& The calculated values should be the same as the live data.  And the efficiency should follow the efficiency diagram XX
		&
		& \\ \hline
	\end{tabular}
	\caption{Something}
\end{table}

\begin{table}[h!]
	\centering
	\label{my-label}	
	\begin{tabular}{|p{1.5 cm}|p{2.1 cm}|p{2.1 cm}|p{2.1 cm}|p{2.1 cm}|}
		\hline
		Test of requirements: 
		& \multicolumn{4}{l|}{Test sensor precision} \\ \hline
		Setup 
		& \multicolumn{4}{l|}{Connect to GUI and power on. No subject! PID is working} \\ \hline
		Test NR:
		& \multicolumn{4}{l|}{3} \\ \hline
		\textbf{Steps} & \textbf{action} & \textbf{Expected result} & 
		\textbf{Actual result} & \textbf{Accept/ Comment} \\ \hline
		1 
		& The terminal to the el motor set a 10-Ohm resistor on. Now on the power terminal to the el motor, supply it with 10V.  
		& Seeing 10V +/-50mV and seeing 1A +/-50mA. 
		&
		& \\ \hline
		2
		& Change the supply to 20V.
		& Seeing 20V +/-50mV and seeing 2A +/-50mA.
		&
		& \\ \hline
		3
		& Turn off supply
		& 
		&
		& \\ \hline
		4
		& Place the subject on the Rolling Road. Start the subject. Use a external RPM counter, and calculate the speed. 
		& The calculate speed should be the same as in the GUI.
		&
		& \\ \hline
		5
		& Lock the axle before between the generator and moment transducer. On the Road (rolling road) mount a newton meter. Pull parallel to it shows 5 newton.  
		& Force in the GUI should be 5 newton +/- XX.
		&
		& \\ \hline
		6
		& Do step 5 again just with 2.5 newton 
		& Force in the GUI should be 2.5 newton +/- XX.
		&
		& \\ \hline
	\end{tabular}
	\caption{Something}
\end{table}


\begin{table}[h!]
	\centering
	\label{my-label}	
	\begin{tabular}{|p{1.5 cm}|p{2.1 cm}|p{2.1 cm}|p{2.1 cm}|p{2.1 cm}|}
		\hline
		Test of requirements: 
		& \multicolumn{4}{l|}{Test max effect on the generator.	PID regulator.} \\ \hline
		Setup 
		& \multicolumn{4}{l|}{Is connect to a GUI and test subject is on the Rolling Road. PID is working.} \\ \hline
		Test NR:
		& \multicolumn{4}{l|}{4} \\ \hline
		\textbf{Steps} & \textbf{action} & \textbf{Expected result} & 
		\textbf{Actual result} & \textbf{Accept/ Comment} \\ \hline
		1 
		& Set Force to 5N and start the subject with a speed of 5m/s  
		& The force should settle in after max 100ms and max peak 5.5N.  
		&
		& \\ \hline
		2
		& Set Force to 25N and change the speed slowly(be carefully) to 8m/s
		& Everything still work. Effect on the GUI should be around 200W.
		&
		& \\ \hline
	\end{tabular}
	\caption{Something}
\end{table}

\begin{table}[h!]
	\centering
	\label{my-label}	
	\begin{tabular}{|p{1.5 cm}|p{2.1 cm}|p{2.1 cm}|p{2.1 cm}|p{2.1 cm}|}
		\hline
		Test of requirements: 
		& \multicolumn{4}{l|}{Dimension.} \\ \hline
		Setup 
		& \multicolumn{4}{l|}{} \\ \hline
		Test NR:
		& \multicolumn{4}{l|}{5} \\ \hline
		\textbf{Steps} & \textbf{action} & \textbf{Expected result} & 
		\textbf{Actual result} & \textbf{Accept/ Comment} \\ \hline
		1 
		& Use a ruler to check dimension.
		& Everything is OK.  
		&
		& \\ \hline
	\end{tabular}
	\caption{Something}
\end{table}

