\newpage
\section{Software PID regulator}

The PID regulator is implemented in code and not in a digital filter block, it's because of a problem that is mentioned in Calculation and simulation section see page \pageref{sec:Calculation_and_Simulation}. The problem is that the "P" part is a function of speed.\\

\subsection{Code}

The mathematical equation for the PID regulator is:\\   

$ PID_{value} \left( s,rpm \right) = {\frac {P_{rpm}\,P}{rpm} \left( 1+{\frac {I}{s}}+ \left( D \right) s \right) } $\\

This is not stable. \\
First. Then the $ rpm $ value goes to zero the $  PID_{value} \left( s,rpm \right) $ will go to infinity. This is prevented with limit the rpm value to minimum 1 rpm.\\
Second. because the solution to generate the torque are with passive components and the speed of the test object. It will not always be possible to reach the wanted torque. If the test object don't have enough speed. This will generate a windup problem. To kill this problem we use back calculation. That will use the previous value that has exceed the limit, and add it to the integral part. 