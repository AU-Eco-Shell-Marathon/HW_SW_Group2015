\newpage
\section{EEPROM memory}
The PSoC used in the Rolling Road has a EEPROM used to save PID, wanted torque and offset values to the sensors.

To make it easier to handle the EEPROM a class has been made. This class only need to be initialized at start up. To initialize it need information about different sizes of data that has to been write down or read from the EEPROM. After that it just to use the write and read function to write and read data from the EEPROM.\\ 
The EEPROM class have a little mechanism to control that where is saved data or not. This will prevent the program from reading old fragmented information.\\
Too see have to use the class goto page \pageref{table:Class_description_EEPROM_RR_PSoC}. \\
To use the EEPROM in the class, two function are used.

\lstset{language=C}
\begin{lstlisting}
#define START_EEPROM_SECTOR  (1u)
#define START_BYTE         ((START_EEPROM_SECTOR * EEPROM_SIZEOF_SECTOR) + 0x00)

cystatus EEPROM_1_WriteByte((uint8)data, START_BYTE + pos);
uint8 EEPROM_1_ReadByte(START_BYTE + pos);
\end{lstlisting}

First function (line 4) are used to write new data to the EEPROM and return a state if it has succeeded or not. The next function (lint 5) are used to get a byte from specific place. START\_BYTE are the start address of the EEPROM. 