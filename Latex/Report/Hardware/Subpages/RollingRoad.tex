\section{RollingRoad} \fxnote{Meningen med det her afsnit er at nævne de vigtige ting i implementeringen, ikke bare ridse de forskellige delemner op, sådan har jeg forstået det hvert fald - TN}
Many of the hardware-designs in Rolling Road has been re-used from the previous design\cite{BAC_rullefelt}. Much of the hardware-section in Rolling Road's documentation\cite{RR} is simply a cut-down version of the previous design where the relevant designs are re-used and calculations are checked for errors. One of the biggest start-up points was that the team had a lot of the earlier components at our disposal and the designs could relatively quickly be built.

\textbf{Voltage Adaptor}\\ 
It was decided that the system's power should be supplied by a standard 12 VDC voltage adaptor. This choice was made in order to minimize the amount of connectors to the system and to make it aesthetically pleasing to look at. Plenty of voltage adaptors were at the teams disposal and the choice was made by analysing the amount of current drawn by the individual components in the system. When connecting the adaptor it will instantly power the individual parts in the systems.

\textbf{Generator}\\
The Generator is used to convert rotational mechanical energy into electrical energy. The generator used in the previous system\cite{BAC_rullefelt} was a 200 W brushless DC-motor \cite{Maxon}. According to the calculations made in the documentation\cite{RR}, this motor was powerful enough to be used to test Zenith33\cite{BAC_zenith33} (and to a later extent, AU2).

\textbf{Torque Sensor}\\
The Torque Sensor is used to measure the test-subject's angular velocity and torque. This could then be used to calculate the subject's mechanical power. Like the Generator, the Torque Sensor was already at our disposal during the start of the project. The sensor is a Contactless Torque Sensor from Lorenz Messtechnik GmbH\cite{TorqueSensor}. Like the Generator, it was found using calculations, that this sensor was good enough to be able to measure the torque correctly in AU2.

\textbf{Signal Converter}\\
The output-signals generated by the torque-measurement in the Torque Sensor was not designed to be read by the Control Unit's specified ADC-range (\SI{0}{\volt} to \SI{5}{\volt}). That meant that the signal had to be rescaled in order to perform correct measurements. This was done using knowledge about operational amplifiers to design a non-inverting adder - which could both rescale and offset the signal. The chosen amplifier is a MCP6004\cite{MCP6004}, which was chosen because a lot of the team-members had experience using this component in prior systems.

\textbf{Level Converter}\\
As not every component in the system could be supplied with 12 VDC there was a need for a Level Converter which could convert this voltage to 5 VDC. The choice fell on the LM7805\cite{LM7805} as a lot of the members in the team had experience using this component in prior systems.

\textbf{Power Sensor}\\
This system measures the amount of electrical power fed into the test-subject's motor. This was done through a LTS 15-NP current transducer\cite{CurrentTransducer}, which measured the current, and a voltage-divider, which measured the voltage. The current transducer had been implemented in the previous system and it was found that it was an easy component to use.

\textbf{Load System}\\ \fxnote{JEg vil sige udvid dette afsnit og genovervej om alle de andre afsnit er nødvendige - TN}
The most complex system in Rolling Road is the Load System. At first glance this system must simply handle the generated electrical power and dissipate it into heat using power resistors\cite{PowerResistor}. The amount of power dissipated could be controlled using a MOSFET\cite{IRFP260N} which switched on and off in order to control the average amount of current flowing into the resistors. However, due to the excessive amount of power generated, the internal reactance in the inductor and the high switching-frequency of the MOSFET some problems arose. These problems came in the form of voltage- and current-transients which had to be reduced in order to get a stable system. This led to the inclusion of a super capacitor\cite{SuperCapacitor} and a snubber-circuit\cite{Snubber}.

\textbf{Control Unit}\\
The Control Unit consists of a CY8CKIT-059 PSoC 5LP Prototyping Kit with an USB-output for easy connection with a PC. This choice was for simplicity due to it being the same micro-controller used in AU2. This controller is also relatively cheap - which proved to be a good thing since a few controllers experienced over-current and died. 

\textbf{Anti-aliasing filter}\\
These filters are implemented between the Control Unit's ADC's and the signals to be measured. As the name hints, these filters are placed in order to dampen frequencies which lie above the Nyquist-frequency (half samplings-frequency). This is done in order to prevent aliasing on the inputs. The filter's are implemented as analog low-pass filters using resistors and capacitors. 

The measurements done by the Power Sensor are filtered through a 2. order low-pass filter. This is also done through two sets of resistors and capacitors - but with a voltage-buffer between them. These voltage-buffers are implemented using operational amplifiers of the type MCP6004\cite{MCP6004}. This choice was made as a DIP-chip\fxnote{Det er vel ligemeget hvilket indpakning ICen er i? - TN} had already been used to implement this component in the Signal Converter - and this chip contains 4 operational amplifiers total.

\textbf{Physical Stand, Control-box and Load-plate}\\
As it is the hope and dream of the developers that Rolling Road can be used for many years ahead, the system was made to allow easy storage and easy transportation. This was done by building the system on as few parts as possible. The Torque Sensor and the Generator had already been mounted on a Physical Stand which in turn had been placed on a wooden frame which holds the test-subject in place. The Physical Stand remained as it was.

Most of the electrical system is build on a designed PCB. This PCB is mounted inside a box (called Control-box) to protect the electrical systems from physical damage. The box contains connectors for easy connection with the other parts of the system.

Due to the size and the generated heat from the Super Capacitor and the Resistors, these components were not placed inside the Control-box. These components were placed on a separate plate (called the Load-plate).

This reduces the amounts of individual parts of the system down to three (plus the computer). The connections between these systems are made user-friendly by using color-coded connectors.