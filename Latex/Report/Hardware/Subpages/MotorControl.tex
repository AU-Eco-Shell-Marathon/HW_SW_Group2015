\section{Motor controller}
In the following paragraphs the hardware implemented for the motor controller will be discussed. 

The hardware developed for the motor controller is realised on a PCB. Doing so requires extra some extra time to realise the designed circuit, as using the ultiboard software requires some adaption. But when then print layout is completed, the implementation is eased as soldering the PCB becomes easier. The downside to using a PCB instead of a breadboard is the added complexity when switching components. That sets some demands for the design to be thouroughly tested before implementing it on a PCB.

Even though implementing the designed circuit on a PCB has it downsides, the plus sides weighs it up. This is because SMD components make the implementation phase better in the sense that they physically are a lot smaller and by that decreases the size of the PCB. Furthermore implementing the circuit onto a PCB gives a series of design features such as minimizing trace to decoupling capacitors. Furthermore it gives the designer a selection amongst many connectors to be used on the PCB. 

 