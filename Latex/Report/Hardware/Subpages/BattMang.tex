\section{Battery Management System}
In the following paragraphs the hardware implemented for the Battery Management System for AU2 will be discussed. 

The Battery Management System consists of the following units:
\begin{itemize}
	\item{Digital Unit}
	\item{Analog Front End}
	\item{Analog Front End Extension Module}
	\item{Isolation Switch}
	\item{DC/DC Converter}
	\item{Current Sensor}
\end{itemize}

When developing the BMS a lot of problems came to light. Some of them required an extensive a mount of troubleshooting. In this section the most essential and time exhausting segments will be discussed. If a more detailed and descriptive troubleshooting process is sought-after then one can be found in the documentation.

The general gist of the problems were that the Analog Front End and the Digital Unit were unable to work in conjunction with of one another. We  were told that the BMS contained a functional CAN interface along with CAN bus - This is how the documentation portrays it. However, when the BMS was tested the CAN bus did not work at all. This led us to believing that software for the BMS was either outdated or the CAN function was commented out. A long search for the code was hereby commenced since the code we had been given was an outdated version from the 2013 BMS Documentation\cite{BMSDocumentation}. 

THIS: In the end we got a hold of the updated code, which was done in 2014 in connection with the 2014 SEM team.

OR: We got a hold of the original developer of the BMS and he was certain that it was the newest source code which we had gotten. We later found the updated code from a member of the 2014 SEM team. 

However, when we were trying to test it an accident occurred which led to a fault on the Analog Front End as well as the Analog Front End Extension Module. This resulted the main battery ICs on both units to fry.
From this moment on many different approaches were taken in trying to resolving the problems with the two units even after the dead ICs were replaced. 
