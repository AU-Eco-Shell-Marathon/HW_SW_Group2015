\chapter{Project formulation}
The purpose of this project has been to create the electrical systems for the car dubbed \textbf{AU2}. The car is designed to compete in Shell Eco-Marathon (SEM) on behalf of AU and should be as energy-efficient as possible. In order to measure and optimize the car's energy-consumption, a dynamometer is developed in parallel with the car's electrical systems. 

The dynamometer is a build on of a previous system with similar functionality\cite{BAC_rullefelt} and much of the design-parameters has been re-used. The new dynamometer has been dubbed \textbf{Rolling Road} and is also part of the project.

In order to control and read the measurements done by Rolling Road, a third system is also being developed in parallel with the two other systems. This system is a GUI and is simply dubbed \textbf{Rolling Road GUI}. This is also part of the project.

In order to compete in SEM the car's electrical systems has to fulfill a number of requirements set down by Shell. Identifying and implementing the solutions to these are part of the project.

Both Rolling Road and AU2 has been designed using former documentations. These documentations are made by both electrical- and mechanical-engineers. Identifying and using the design-parameters from these documentations has also been part of this project.

As of the 24\textsuperscript{th} of May 2016 it has become clear that AU - Shell Eco Marathon Team isn't able to meet the deadline and will not be able to compete in Shell Eco-Marathon 2016. Had the team met the deadline and were able to compete in the SEM, the following points had also been a part of the project:
\begin{itemize}
	\item Complete the electrical systems in AU2 in order to guarantee full functionality during Shell Eco-Marathon and installing the systems in the car.
	\item Having complete backup-circuits of every electrical system in the car.
	\item Creating a technical documentation which met the requirements set down by Shell.
	\item Optimizing the car's driving-algorithm in order make the car as efficient as possible.
	\item Equipping the pit area with the necessary tools in order to maintain the car on site, during Shell Eco-Marathon.
	\item Organizing the trip to London, lodging of the participants during Shell Eco-Marathon and other necessary points.
	\item Organizing the team's duties on site during Shell Eco-Marathon.
\end{itemize}
Many of these points had future deadlines than the 27\textsuperscript{th} of May 2016 but were nonetheless considered problems which must have been solved in order to compete in SEM.