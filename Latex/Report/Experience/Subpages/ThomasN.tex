\section{Thomas B. Nielsen}
I have learned a great during this project. The learning have both consisted of fundamental group processes and technical skills. 

As this project was different from a normal semester project, every phase of the project has deviated from the norm.  The entire project has rested on the ability to comprehend the large amount of informations already made available by the previous works on the Shell Eco Marathon. This meant a greal deal of reverse engineering, which isn't learnt in school and therefore required some extra time to get used to. 

Throughout this project there was a great need of communication. Both internally in the semester group, but also across the studies with the mechanical engineers. The need for communication in this project has been clear. Both in the sense of quality and quantity, meaning that there should have been both more and better communication with the mechanical engineers. 

The technical experiences I have received in this project are widespread. The most useful I will take with me is the ability to choose and select components based on the requirement for the it and the information made available through datasheets. There has been a great deal of component selection for the motor controller and the market consists of an endless list of solutions. Furthermore I have gained experience in the art of testing the produced hardware. This includes the most optimal test cases and generally being sceptical of the test results. In addition to the aforementioned I have received increased knowledge of the software at use in designing a PCB and in general PCB layout. 