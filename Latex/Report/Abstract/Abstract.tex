\chapter{Abstract}
This report describes the 4\textsuperscript{th} semester project E4PRJ4 at Aarhus University School of Engineering. The study deals with three major systems; a dynanometer named Rolling Road, the electrical systems of Aarhus University's vehicle dubbed AU2 and an user interface called Rolling Road GUI. The main goal of the project is to complete and prepare the systems to take part in the Shell Eco-Marathon, an international competition pertaining to the submission of the most fuel efficient vehicles. 
%Nonetheless, in the documentation each of the systems are comprehensively described in detail. This includes, however, is not limited to the systems' documentations of software and hardware. Furthermore, the working process has also been addressed from beginning to end \fxnote{Omformuler - TN}. 

AU2 contains a vast variety of capabilities and can among other things utilize lithium polymer batteries to power the complex systems contained in AU2.
Protection of the batteries is enforced with a Battery Management System. Furthermore, a highly energy-efficient driving algorithm is used to ensure optimal driving operation, which is enforced by the Motor Control System. Data can also be logged to be analysed after a run on a track.
The Rolling Road and the Rolling Road GUI can be used together or separately to test the efficiency of the drive train of AU2 and optimize it for a given run or track. The GUI offers a real time overview of the test results from Rolling Road simulations.
The development process is completed with the use of an suitable adaptation of SCRUM, where each week consisted among other things of sprint coordination, stand-up - and consultant meetings.

The Rolling Road and Rolling Road GUI are both implemented and working, however, some parts of AU2 are not functioning as a result of the shell of AU2 not being finished in time.