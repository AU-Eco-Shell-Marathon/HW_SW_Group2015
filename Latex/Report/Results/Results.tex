\chapter{Results}
AU2 has not attained an acceptable result because of errors in the Battery Management System. To solve these problems a lot of debugging had already been performed, though to no avail. Possible approaches at this stage would have been to buy a commercially available BMS which fulfilled the requirements (such as a battery-cell count of 12 or more). If such a BMS could not be found and certain requirements weren't met by the bought BMS then those would have to be implemented to be able to compete in the SEM\fxnote{omformuler der her: hvis i ikke havde fundet et BMS så blev i nød til at bruge det i ikke have? - TR}. That would have been a very costly option. However, another solution would have been to hire the original developer of the BMS 2013 to consult the current BMS-team on the matter. These approaches were chosen to be evaluated after the hand-in of the project documentation and report. However, since participation in the SEM 2016 was cancelled none of these options were carried out. 

The Rolling Road system works in close collaboration with the GUI and therefore most of the accept test will work either directly in correlation with the Rolling Road or the GUI will be a part of the test. \fxnote{omformuler: vil accepttesten virke eller hvad? - TR}This is because of the entanglement between the two systems. Furthermore the GUI is developed for direct use with the Rolling Road and therefore many of the requirements set forth for the Rolling Road will be adjustable by the GUI. The general functions of the Rolling Road are fully developed which can be seen by the number of the test cases completed in the accept test.

The Rolling Road GUI has reached a satisfying degree given the fact that it works seamlessly with or without the Rolling Road. For testing there has been developed a series of tests, in total 189 unit-tests giving a code coverage procent $\sim$75\%, this verifies most of the written software is working as intended. \fxnote{omformuler - TR}

