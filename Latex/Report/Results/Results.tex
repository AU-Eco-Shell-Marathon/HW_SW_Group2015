\chapter{Results}
AU2 has not attained an acceptable result because of errors in the Battery Management System and Motor Control System. To solve these problems a lot of debugging had already been performed, though to no avail. Possible approaches at this stage would have been to buy a commercially available BMS which fulfilled the requirements (such as a battery-cell count of 12 or more). If such a BMS could not be found and certain requirements weren't met by the bought BMS then those would have to be implemented into the bought BMS to be able to compete in the SEM. That would have been a very costly option. However, another solution would have been to hire the original developer of the BMS 2013 to consult the current BMS-team on the matter. These approaches were chosen to be evaluated after the hand-in of the project documentation and report. However, since participation in the SEM 2016 was cancelled none of these options were carried out.
The same is true for the Motor Control System. A lot of work would have been placed into making a working PCB, so it would be able to compete in SEM. But due to the cancellation of participation in SEM, the time required to make the PCB funcional was managed otherwise.

Nonetheless, most of the accept test have failed due to the car not being physically done and therefore it was not possible be perform the accept test properly.

The Rolling Road system works in close collaboration with the GUI and therefore the GUI is an integral part of the accept test. This is because of the entanglement between the two systems making the GUI and Rolling Road co-depended. Furthermore the GUI is developed for direct use with the Rolling Road and therefore many of the requirements set forth for the Rolling Road will be adjustable by the GUI. The general functions of the Rolling Road are fully developed which can be seen by the number of the test cases completed in the accept test.

The Rolling Road GUI has reached a satisfying degree given the fact that it works with or without the Rolling Road and passed all requirements except one. To test this there has been developed a series of tests, in total 189 unit-tests. All these tests are passed, but yields a code coverage percent of $\sim$75\% . This verifies that the 75\% is working as intended and the last 25\% is unaccounted for.

