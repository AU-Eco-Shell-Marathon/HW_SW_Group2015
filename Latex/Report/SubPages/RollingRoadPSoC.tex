\section{Rolling Road PSoC}
The Rolling Road function is to test a test-objects efficiency, with a specify torque or force. To do this it has to measure both the mechanic and electric power usage. From this the efficiency can be found.\\

It has to make it easy to use from a control unit \vref{Rolling_Road_GUI} through a serial COM connection. The control unit can send a list of commands, to change PID parameter, change wanted torque and calibrate sensors. The Rolling road will also feedback information from all sensors, and calculation of power and efficiency.\\
\subsection{Design}
The rolling road is made in c but is written in a more object form to look like c++. Because it is easier to design it as a objective program. To design the solution, a class diagram was made see Rolling road doc\fxnote{lav reference}.\\

The design of Rolling Road consist of three things: regulating, measure and communication. Where all these work together.\\

The reason that it is programmed in c and not c++ are because PSoC creator don't have a build in c++ compiler. 
\subsection{Implementation}
In the implementation another problem was made clear. measure of sensors and the calculation are very time critical. To get a better overview over the flow, a flow-diagram was made see \vref{fig:data_flow_diagram}.\\

This flow-diagram was used to design flow of the data, and there the calculation should happen. As seen in the diagram the regulator gets the value faster then the control unit. Because to big time delay would made the system unstable.\\

the different signals from the sensors goes though a couple of steps before they are calculated and passed on. To save as much CPU time as possible, DMA was setup to make the flow use less CPU time. This is done on the torque sensor. Because it has be measured with as little time delay as possible. Where are implemented some decimation filter in the design, because the sensor are measured with a big sample frequency around 50 kSample/s. Where the PID regulator only runs with 2 kSample/s and the data transmitting to the control unit only run with a speed of 2-100 Sample/s.\\
There is also implemented a small exponential moving average filter in c what will smooth all data before sending to the control unit.    
\begin{figure}[H]
	\centering
	\includegraphics [width=6in]{../Documentation_RR/Software/Pictures/data-flow.png}
	\caption{Rolling Road data-flow diagram.}
	\label{fig:data_flow_diagram}
\end{figure}
\subsection{Test}
To test Rolling Road a test stand was setup with a test car figure \ref{fig:RR_first_test}. Where the measure of the car was made. 
\begin{figure}[H]
	\centering
	\includegraphics [width=3in]{SubPages/Images/jens_test.png}
	\caption{First test of the old AU car.}
	\label{fig:RR_first_test}
\end{figure}
After a half hour test a efficiency diagram could be made from the data \ref{fig:RR_first_test_result}.
\fxnote{compiler ikke}
%\begin{figure}[H] 
%	\centering
%	\includegraphics [width=3in]{SubPages/Images/RR_test_result.eps}
%	\caption{efficiency diagram of the first test, the value is in \% efficiency.}
%	\label{fig:RR_first_test_result}
%\end{figure}
This result of course is not close enough to the efficiency what was calculated by machine-engineer. But the car is not in the best sharp, and it is not fasten to the Rolling Road. So there is many places there is efficiency loose.


