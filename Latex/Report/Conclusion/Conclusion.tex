\chapter{Conclusion}
The overall conclusion for the project must be that it wasn't as successful as hoped. Many items was left unfinished and there was a number of assignments that had yet to be begun. The problems mainly revolved around 2 things. First off, blocks that was supposed to work and had worked before, was in fact lacking a lot of work before it would be pass the intern quality assurance. Second off, the collaboration with the mechanical engineers was a greater time consumer than expected and induced a number of problems.   

The electrical system for the car was designed, but was not thoroughly tested and stress tested. Most of the software have been written, but yet to be integration-tested because of the implications with the hardware. There is a lot of work left in this section. The driving algorithm had not been tested at all and therefore no optimization on this has been done by the hand-in data. Herein lies a great deal of work as well. The program itself developed for the motor controller unit has been unit tested with success and it is expected to be fully functional. 

The battery management system(BMS) is at hand-in not functional. There has been performed extensive tests on this and the problem was not to be found. This was due to changes to hardware, that was left undocumented. The software following the BMS is expected to be fully functional. This expectation is made on the basis of the BMS being fully operationel during previous SEM.

The Rolling Road became operational and was running successful. Rolling Road was implemented in such a manner that it will be easy for new groups to adapt to using the product. 

The Rolling Road GUI was developed and at the end became an almost finished product.

A great deal of this project was to read and understand documents from various sources. It was especially needed to read and comprehend the large amount of documentation made beforehand and use this knowledge to move forward. Furthermore, picking out the requirements for the electrical system set forth by Shell was made as well.

Lastly it was chosen not to compete in the Shell Eco Marathon, as the shell for AU2 would not be completed in time. It is with great regret that this team sees a huge workload go to waste. This also means a lot of the work with the practical part of going to London will not be initiated and the experience the team would have gotten by being a part of that will not be achieved.
