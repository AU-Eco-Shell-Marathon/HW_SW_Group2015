\chapter{Conclusion}
The overall conclusion for the project is that it isn't as successful as the team had hoped. Many items were left unfinished and there were a number of assignments which had not yet entered their development-phase. The problems mainly revolved around two things. First off: Blocks which were delivered to the team at the start of the project and were supposed to work, were in fact not working. These blocks required a lot of work before it would be able to pass the internal quality assurance. Second off: The collaboration with the mechanical engineering students wasn't very great because the mechanical team was poorly organised after their project leader had to leave the mechanical team.

The electrical system for the car was designed, but was not thoroughly tested nor stress tested. Most of the software have been written, but have yet to be integrated because of the limitations made by the hardware. There is a lot of work left to be done in this area. The driving-algorithm had not been tested at all and at the time of this hand-in, no optimization of the car had been made. The optimization contains a lot of work as well. The program which was developed for the Motor Controller Unit has been unit tested with success and it is expected to be fully functional and should simply be a plug-and-play feature for the next group working on the project. 

The Battery Management System (BMS) is at hand-in not functional. There has been performed extensive tests on this and the problem was not found. This was due to changes to hardware, that was left undocumented by the previous developers. The software in the BMS is expected to be fully functional. This expectation is made on the basis of the BMS being fully operational during SEM 2014 where it was implemented and used in Zenith33.

Rolling Road became operational and was running successfully at the time of the hand-in. The blocks in Rolling Road were implemented in such a manner that it will be easy for new groups to start using the system. 

The Rolling Road GUI was developed and at the end became a working product.

Much of the workload in this project lay in the fact that the group had to read and understand previous documentations about various systems. The knowledge gained from these documentations was used to move forward in this project. Not every decision made by previous groups could be re-used due to the fact that the requirements needed for participation in SEM 2016 had been altered since SEM 2014. A lot of work laid in finding the requirements relevant for designing the electrical systems since the requirements were made external.

During the final phase of the project it was decided that the team would not compete in SEM 2016 as the team of mechanical engineering students would not be able to complete the frame of AU2 in time. This decision was met with great regret as the team will not be able see their work pay off in the form of participation in SEM 2016.