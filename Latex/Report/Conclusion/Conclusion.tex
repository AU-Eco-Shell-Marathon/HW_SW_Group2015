\chapter{Conclusion}
The overall conclusion for the project is that it isn't as successful as the team had hoped. Many items were left unfinished and there were a number of assignments which had not yet entered their development-phase. The problems mainly revolved around two things. First off: Blocks which were delivered to the team at the start of the project and were supposed to work, were in fact not working. These blocks required a lot of work before it would be able to pass the internal quality assurance. Second off: The collaboration with the mechanical engineers wasn't very well organized and most of the time the meetings didn't lead anywhere.   

The electrical system for the car were designed, but was not thoroughly tested nor stress tested. Most of the software have been written, but have yet to be integrated because of the limitations of the hardware. There is a lot of work left to be done in this area. The driving-algorithm had not been tested at all and at the time of this hand-in, no optimization of the car had been made. The optimization contains a lot of work as well. The program which was developed for the Motor Controller Unit has been unit tested with success and it is expected to be fully functional and should simply be a plug-and-play-feature for the next group working on the project. 

The Battery Management System (BMS) is at hand-in not functional. There has been performed extensive tests on this and the problem was not to be found. This was due to changes to hardware, that was left undocumented. The software following the BMS is expected to be fully functional. This expectation is made on the basis of the BMS being fully operationel during previous SEM.

The Rolling Road became operational and was running successful. Rolling Road was implemented in such a manner that it will be easy for new groups to adapt to using the product. 

The Rolling Road GUI was developed and at the end became an almost finished product.

A great deal of this project was to read and understand documents from various sources. It was especially needed to read and comprehend the large amount of documentation made beforehand and use this knowledge to move forward. Furthermore, picking out the requirements for the electrical system set forth by Shell was made as well.

Lastly it was chosen not to compete in the Shell Eco Marathon, as the shell for AU2 would not be completed in time. It is with great regret that this team sees a huge workload go to waste. This also means a lot of the work with the practical part of going to London will not be initiated and the experience the team would have gotten by being a part of that will not be achieved.
