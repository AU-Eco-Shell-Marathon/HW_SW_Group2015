\input{Preamble}
\newcommand{\HRule}{\rule{\linewidth}{0.1mm}}

\begin{document}
	\begin{center}
		{\Huge \bfseries \textsc{Samarbejdsaftale}}\\
		\textsc{\large 4. semesterprojekt - Gruppe 1}\\[0.3cm]
	\end{center}
\hfill
	\begin{itemize}
		\item Møder:
			\begin{itemize}
				\item Gruppemøder med vejleder holdes en gang om ugen
				\begin{itemize}
				\item Mødelængden reguleres efter vurdering.
				\item Afbud skal ske seneste 1 dag før mødedato.
				\item For specielle årsager, så som ulykker e.l. accepteres senere afbud.
				\item Mødeindkaldelse, dagsorden og referat fra sidste møde skal seneste sendes ud på E-mail til deltagerne 1 dag før mødet.
				\item Har man ønsker til punkter på dagsordnen skal de sendes til Jens Peter Nymann.
				\end{itemize}
			\end{itemize}
	
		\item Der vil være møder dagligt/de dage der arbejdes
			\begin{itemize}
				\item Disse møder aftales internt uden mødeindkaldelse. 
				\item De daglige møder fungerer uden vejleder.
			\end{itemize}
		
		\item Referat:
			\begin{itemize}
			\item Laimonas er referent på hvert vejledermøde, med mindre dette ikke er muligt.
			\item Referat skal lægges op på GitHub senest dagen efter mødet.
			\item Referatet skrives ud fra templaten som ligger på GitHub.
			\end{itemize}
		\item Aftaler:
			\begin{itemize}
			\item Mundtlige aftaler er ligestillet med skriftlige aftaler, og skal derfor overholdes.
			\item Det forventes, at folk møder til aftalt tid.
			\item Overholdes aftaler ikke, eller man ikke kommer til møderne tages dette internt i gruppen. Kan det ikke løses internt kontaktes vejleder. 
			\end{itemize}
		\item Beslutninger:
			\begin{itemize}
			\item Beslutninger tages ved demokratisk flertal.
			\item Er der 3 for og 3 imod, og der på ingen måde kan findes en løsning kontaktes vejleder.
			\end{itemize}
		\item Roller:
			\begin{itemize}
				\item Jens fungerer som projektleder, hvis funktioner er at holde styr på:
				\begin{itemize}
					\item Datoer/review
					\item Struktur
				\end{itemize}
				\item Jonas fungerer som scrum master på de daglige møder uden vejleder.
				\item Thomas Rasmussen står for mappestrukturen og at sikre samling af dokumenter, samt versioner af disse.
				\item Jens fungerer som mødeleder
				\begin{itemize}
					\item Mødelederen står for mødeindkaldelse, samt ordstyrer rollen ved det ugentlige vejledermøde. 
				\end{itemize} 
			\end{itemize}
		\item Pipeline
			\begin{itemize}
				\item GitHub bruges som filtjeneste.
				\item Rapport og dokumentation mm. skrives i LaTeX.
				\item Version 13 af multisim bruges til dette projekt.
				\item Microsoft Visio 2013 bruges til arkitektur.
				\item PSoC Creator 3.3 bruges til programmering af PSoC.
				\item Microsoft Visual Studio 2015 til programmering af GUI.\\
			\end{itemize}
		\item Officielle kommunikationskanaler er:
		\begin{itemize}
			\item	E-mails
			\item 	Facebook gruppen (4. Semester Projekt F2016)
		\end{itemize}
	\end{itemize}


\end{document}