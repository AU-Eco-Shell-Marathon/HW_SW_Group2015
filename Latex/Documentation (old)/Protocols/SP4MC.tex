
\section{SP4MC - Serial Protocol for MotorControler}
Each command is termintad by a newline character, and each value in a command is separated by a whitespace.

The following table assumes that the controller is using Coast-n-Burn.

Overview of packets:
\begin{table}[h!]
	\centering
	\label{Protocol:overviewMC}
	\begin{tabular}{l|llll}
		\# & Description 		& Command         & Direction             & Example     		\\\hline
		0  & Handshake   		& MotorController & PC $\rightarrow$ PSoC & 0 MotorController 	\\
		1  & Set max speed 	    & <int>           & PC $\rightarrow$ PSoC & 1 30 				\\
		2  & Set cruise speed	& <int>           & PC $\rightarrow$ PSoC & 2 25       			\\
	\end{tabular}
	\caption{Overview for commands sent between PSoC og PC}
\end{table}

As with the protocol used for RollingRoad, this protocol will have a similar structure. Where the first number send, tells the PSoC, what kind of informations is expected here. Then the actual information is send. 

\subsection{Handshake}
The Handshake is used to ensure that the PC is connected correctly to the PSoC. The PSoC must respond with same command in order for the connection to be opened.

\subsection{Max speed}
The set max speed command is used to set the maximum allowed speed for the motor to be driven with. 

The maximum allowed value is 35. The minimum value must be 20. The unit here being km/h.

\subsection{Cruise speed}
The set cruise speed command sets the value at which the vehicle will cruise with. Under the assumption that Coast-n-Burn is used, this will be average speed of the vehicle. This means that vehicle will fluctuate around this value.

The maximum value here is defined by previously set Max speed. The cruise speed must be at least 3 km/h smaller than what the Max speed is set at. The minimum value must be 10. The unit here being km/h.


