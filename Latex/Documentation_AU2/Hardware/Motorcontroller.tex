\section{Motorcontoller}
The purpose of the Motorcontroller is to adjust the speed of the vehicle. In this design it is done by a PWM-signal and a MOSFET. It is of the utmost importance that the hardware affects the motor in the least possible way and thereby affecting the efficiency of the motor itself.

Aside from controlling the motor, this circuitry also senses the voltage and current consumption of the motor. This data is going to the PSoC and is here analyzed. There is, amongst other things, an overcurrent protection built in, to secure the PCB. For more information on this subject see \vref{sec:measurements}.

\subsection{Design}
There are several design considerations to take into account when designing this motorcontroller. The primary and most important feature to this design is the MOSFET that is used. Furthermore the resistor placed between the driver and the MOSFET, decides the amount of current going into the MOSFET and by that how fast it switches(more on this later). On figure \vref{Motorcontroller} it is dubbed 'R18'. 

The figure below(\vref{Motorcontroller}) shows a snippet of the motorcontroller. The parts left out are the measurements as discussed in \vref{sec:measurements}.

There are three components at use here:

\begin{itemize}
	\item{IRFB7530 = Power MOSFET}
	\item{MBR20200CT = Power Diode}
	\item{MCP1407 = MOSFET Driver}
\end{itemize}

The design considerations when using these specific components will be discussed here. Furthermore the components specific application purpose will be mentioned. Lastly, it will be explained as to why this component was chosen.  

\textbf{IRFB7530}

This com

\textbf{MBR20200CT}

This diode is a flyback diode, in place to secure a known passage for the generated current by the motor, when the MOSFET is shut off. When the motor is supplied is supplied with a current it works as a motor, but when the current is shut off, the motor works as a generator. It is necessary to secure components from possible breaking down after trying to sink the current from the generated motor. The diode is placed in parallel with the motor.

A schottkey diode is chosen for its very fast switching time and less for the low voltage drop. This means that there is a smaller power dissipation in the diode and thereby less loss which equals a higher efficiency. 

This diode is chosen for its specifications to be able to withstand the current coming from the motor. It can be switched with another diode with similar specifications.

\textbf{MCP1407}

This component was chosen as it was already implemented in the Rolling road design and therefore the group had both experience and spare parts of this. This component is easily interchangeable with other components of the same specifications. 

\begin{figure}[H]
	\centering
	\includegraphics[width=0.85\linewidth]{Hardware/Pictures/Motorstyring}
	\caption{Motor controller hardware}
	\label{fig:Motorcontroller}
\end{figure}


\subsection{Unity test}
text