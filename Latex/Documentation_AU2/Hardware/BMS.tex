\section{Battery Management System}
The purpose of the Battery Management System (BMS)...

\subsection{Design}
text

\subsection{Implementation}
text

\subsection{Unity test}
text

\subsection{Debugging}
The purpose of this section is meant to help whoever is going to work with the BMS in the future. Here you will be guided in how to use, debug and program the BMS.

\subsubsection{Programming the BMS}
Firstly, you will need a few prerequisites listed below to be able to program the BMS. Since the BMS was originally build in 2013 the software to program the BMS is outdated but sometimes it is only possible to proceed and get it to work with the specified utilities. 

Tools required to program the BMS are listed here:
	\begin{itemize}
		\item AVR Studio 4 (tested with version 4.19 Build 730).
		\item Atmel Studio 6 or 7 (works with both).
		\item JTAG-ICE programmer	(programmer that is included with the BMS hardware).
		\item 2 batteries to power the BMS (used to power the BMS).
		\item Tera Term (or any other hyperterminal).
		\item USB-A to USB mini cable	(used to debug the BMS and see readouts from Tera Term).
	\end{itemize}
After you have acquired these tools you are ready to proceed. First of you connect the USB cable from your computer to the BMS frontend. Then you launch your chosen hyperterminal and select the COM-port which the BMS has. Afterwards you configure the hyperterminal, where the two most important settings are the buad rate and the way you receive from the BMS. Set the baud rate to 38400, 1 stop bit, 8 data bits and 0 parity bits. Then you specify the newline settings so that you receive both CR+LF else select AUTO. In Tera Term you go to Setup->Terminal->New-line. 
Now you're ready to receive data from the frontend. 