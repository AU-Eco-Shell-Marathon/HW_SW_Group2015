\section{Functional requirements}
The functional requirements for AU2 are labeled and specified in the table below.

\begin{table}[h!]
	\label{FREQ_AU2}
	\centering
	\begin{tabular}{|p{2 cm}|p{7 cm}|p{4 cm}|}
		\hline
		\textbf{Req. \#} & \textbf{Description} & \textbf{Comments} \\\hline
		AU2\_F1		& The car must have one electric storage device, one motor and one control unit. &   \\\hline
		AU2\_F2		& The car must be equipped with a built-in horn mounted towards the front of the vehicle, which can be activated by the driver in the cockpit. &   \\\hline
		AU2\_F3		& The car must be equipped with an emergency shutdown mechanism, which must isolate the battery from the propulsion system when the button is pressed. &   \\\hline
		AU2\_F4		& The car must include a "dead man's safety switch" which must be activated at all time in order for the car to drive. &   \\\hline
		AU2\_F5		& The battery must be placed on a metal tray in order to prevent an eventual battery-fire to damage the system. &   \\\hline
		AU2\_F6		& The positive and negative circuits of the propulsion battery and super capacitors must be electrically isolated from the vehicle frame. &   \\\hline
		AU2\_F7		& The battery, all electrical circuits and the super capacitors must be protected against electrical overload. &   \\\hline
		AU2\_F8		& The battery must be installed outside the cockpit behind a bulk head. &   \\\hline
		AU2\_F9		& The car must be equipped with a joule-meter located between the battery and the motor controller. The display must be readable from outside the vehicle's body. &   \\\hline
		AU2\_F10	& The car should be equipped with a data-log, which is able to collect data concerning the power from the battery and the wheel's angular velocity. &   \\\hline
		AU2\_F11 \fxnote{Holder vi fast i dette?}	& The car should be equipped with a speedometer with a display, which is readable from inside the cockpit. &   \\\hline
	\end{tabular}
	\caption{Functional requirements concerning AU2}
\end{table}