\section{Functional requirements}
The functional requirements for AU2 are labeled and specified in the table below.

\begin{longtable}{|p{2 cm}|p{7 cm}|p{4 cm}|}
		\hline
		\textbf{Req. \#} & \textbf{Description} & \textbf{Comments} \\\hline
		AU2\_F1		& The car must have one electric storage device, one motor and one control unit for the motor. &   \\\hline
		AU2\_F2		& The car must be equipped with a built-in horn mounted towards the front of the vehicle, which can be activated by the driver in the cockpit. &   \\\hline
		AU2\_F3		& The car must be equipped with an emergency shutdown mechanism, which must isolate the battery from the propulsion system when the button is pressed. &   \\\hline
		AU2\_F4		& The car must include a "dead man's" safety switch which must be activated at all times in order for the car to drive. &   \\\hline
		AU2\_F5	& The car must be equipped with spade connectors that fit a joulemeter, which is to be located between the battery and the Motor Controller System(MCS). The display must be readable from outside the vehicle's body. &   \\\hline
		AU2\_F6	& The car should be equipped with a data-log, which is able to collect data concerning the power from the battery and the wheel's angular velocity. &   \\\hline
		AU2\_F7	& The car must be equipped with a speedometer. &   \\\hline
		AU2\_F8	& The car's BMS must be able to protect the cells of the battery from undervoltage and overvoltage. This requirement is already implemented in the BMS from earlier projects. (BMS\_F.1)\cite{BMSDocumentation} (section 1.5.1.2). &   \\\hline
		AU2\_F9	& The car's BMS must be able to protect the battery from emitting overcurrent. This requirement is already implemented in the BMS from earlier projects. (BMS\_F.1)\cite{BMSDocumentation} (section 1.5.1.2). &   \\\hline
		AU2\_F10	& The car's BMS must be able to balance the cells if needed. This requirement is already implemented in the BMS from earlier projects. (BMS\_F.4)\cite{BMSDocumentation} (section 1.5.1.2). &   \\\hline
		AU2\_F11	& The car's BMS must be able to protect the battery from overheating. The threshold when discharging is 60 degrees celsius maximum. And when charging, the threshold is 45 degrees celsius maximum. This requirement is already implemented in the BMS from earlier projects. (BMS\_NF.2)\cite{BMSDocumentation} (section 1.5.2.2). &   \\\hline
		AU2\_F12	& The car's BMS should be able to transfer data to the MCS via CAN-communication so the data can be collected on a SD-card. This requirement is already implemented in the BMS from earlier projects. (BMS\_F.8)\cite{BMSDocumentation} (section 1.5.1.2). &   \\\hline
\end{longtable}