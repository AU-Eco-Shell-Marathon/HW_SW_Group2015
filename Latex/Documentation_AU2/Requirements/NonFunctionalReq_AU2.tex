\section{Non-functional requirements}
The non-functional requirements for AU2 are labeled and specified in the table below\cite{ShellRequirements}.

\begin{table}[h!]
	\centering
	\label{my-label}
	\begin{tabular}{|p{2 cm}|p{7 cm}|p{4 cm}|}
		\hline
		\textbf{Req. \#} & \textbf{Description} & \textbf{Comments} \\\hline
		AU2\_NF1	& The horn must emit a sound equal or louder than 85 dBa when measured 4 meters horizontally from the vehicle. &	\\\hline
		AU2\_NF2	& The tone emitted by the horn must have a pitch between 420 Hz and 420 kHz. &	\\\hline
		AU2\_NF3	& The voltage from the battery must not exceed 48 V nominal and 60 V maximum. &	\\\hline
		AU2\_NF4	& The capacity of the battery must not exceed 1,000 Wh. &	\\\hline
		AU2\_NF5	& The Motor Control System must be able to measure AU2's velocity within a 0-30 km/h range and with a precision of $\pm$2 km/h. &	\\\hline
		AU2\_NF6	& There must be backups for all PCBs used in AU2. And at least two sets of batteries and motors. &	\\\hline
		AU2\_NF7	& The battery must be placed on a metal tray in order to prevent an eventual battery-fire that would damage the system. &   \\\hline
		AU2\_NF8	& The positive and negative circuits of the propulsion battery must be electrically isolated from the vehicle frame. &   \\\hline
		AU2\_NF9	& The battery must be installed outside the cockpit behind a bulk head. &   \\\hline
		AU2\_NF10	& The threshold when discharging is 60 degrees celsius maximum &   \\\hline
		AU2\_NF11	& The threshold when charging is 45 degrees celsius maximum. &   \\\hline
	\end{tabular}
	\caption{Non-functional requirements concerning AU2}
\end{table}
