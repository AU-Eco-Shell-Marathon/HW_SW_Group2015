\pagebreak
\section{Test of Non-functional requirements}

\begin{longtable}{|p{1.7 cm}|p{2.9 cm}|p{2.9 cm}|p{1.9 cm}|p{1.8 cm}|p{1.8 cm}|}
	\hline
	\textbf{Req. \#} & \textbf{Description} & \textbf{Test procedure} & 
	\textbf{Expected result} & \textbf{Actual result} & \textbf{Accept/ Comment} \\ \hline
	\endhead
	AU2\_NF1 
	& The horn must emit a sound equal or louder than 85 dBa when measured 4 meters horizontally from the vehicle.
	& The horn button is pressed and the sound pressure level is measured using a decibel-meter.
	& The decibel-meter shows a value above 85 dBa.
	& The decibel-meter shows a value at 71 dBa.
	& Failed.\\ \hline
	AU2\_NF2
	& The tone emitted by the horn must have a pitch between 420 Hz and 420 kHz.
	& The horn-button is pressed and the frequency is measured using an electronic tuner.
	& The measured tone has a frequency which lies between 420 Hz and 420 kHz.
	& The electronic tuner measures a frequency of 2.5 kHz.
	& Success.\\ \hline
	AU2\_NF3
	& The voltage from the battery must not exceed 48 V nominal and 60 V maximum.
	& A multimeter is used to measure the maximum voltage of a fully charged battery. 
	& The measured voltage is not above 60 V.
	& The measured voltage is 49.8 V.
	& Success.\\ \hline
	AU2\_NF4
	& The capacity of the battery must not exceed 1,000 Wh.
	& The capacity of the battery is measured with a Electronic Load.
	& The Electronic Load shows a value below 1,000 Wh.
	& This test was done in the previous documentation of BMS2013\cite{BMSBatteryTest}.
	& Success.\\ \hline
	AU2\_NF5
	& The Motor Control System must be able to measure AU2's velocity within a 0-30 km/h range and with a precision of $\pm$2 km/h.
	& AU2 is placed on the Rolling Road to measure the velocity. This velocity is compared to the internally measured velocity on AU2.
	& The measured speed is identical to the vehicles speed.
	& The body of AU2 is not completed. Therefore the test could not be executed successfully.
	& Failed.\\ \hline		
	AU2\_NF6
	& There must be backups for all PCBs used in AU2. And at least two sets of batteries and motors.
	& Inspect the PCBs, their backup versions, the sets of batteries and motors, to insure that there are two working sets of each.
	& There are two working versions of each PCB, battery-pair and motor.
	& There are not two working versions of each PCB, battery-pair and motor.
	& Failed.\\ \hline
	AU2\_NF7	& The battery must be placed on a metal tray in order to prevent an eventual battery-fire that would damage the system.
	& AU2 is inspected in order to determine if the battery is placed on a metal tray.
	& The battery is located on the metal tray in AU2.
	& The battery is located on the metal tray, but not in AU2. 
	& Failed.\\ \hline
	AU2\_NF8 
	& The positive and negative circuits of the propulsion battery must be electrically isolated from the vehicle frame.
	& AU2 is inspected in order to determine if the electrical systems are isolated from the frame of AU2.
	& The systems are isolated from the frame of AU2.
	& The body of AU2 is not completed. Therefore the test could not be executed successfully.
	& Failed.\\ \hline
	AU2\_NF9 
	& The battery must be installed outside the cockpit behind the bulk head.
	& AU2 is inspected in order to determine the location of the battery.
	& The battery is isolated from the driver's compartment and behind the bulk head. 
	& The body of AU2 is not completed. Therefore the test could not be executed successfully.
	& Failed.\\ \hline
	AU2\_NF10
	& The threshold when discharging is 60 degrees celsius maximum. 
	& 
	& \\ \hline
\end{longtable}