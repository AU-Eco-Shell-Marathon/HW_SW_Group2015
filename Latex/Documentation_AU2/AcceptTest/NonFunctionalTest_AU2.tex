\pagebreak
\section{Test of Non-functional requirements}

\begin{longtable}{|p{1.7 cm}|p{2.9 cm}|p{2.9 cm}|p{1.9 cm}|p{1.8 cm}|p{1.8 cm}|}
	\hline
	\textbf{Req. \#} & \textbf{Description} & \textbf{Test procedure} & 
	\textbf{Expected result} & \textbf{Actual result} & \textbf{Accept/ Comment} \\ \hline
	\endhead
	AU2\_NF1 
	& The horn must emit a sound equal or louder than 85 dBa when measured 4 meters horizontally from the vehicle.
	& The horn button is pressed and the sound pressure level is measured using a decibel-meter.
	& The decibel-meter shows a value above 85 dBa.
	& 
	& \\ \hline
	AU2\_NF2
	& The tone emitted by the horn must have a pitch between 420 Hz and 420 kHz.
	& The horn-button is pressed and the frequency is measured using an electronic tuner.
	& The measured tone has a frequency which lies between 420 Hz and 420 kHz.
	& 
	& \\ \hline
	AU2\_NF3
	& The voltage from the battery must not exceed 48 V nominal and 60 V maximum.
	& A multimeter is used to measure the maximum voltage of the battery. And note that the nominal voltage of the battery is 3.7V per cell, there is 12 cells in the battery, where the maximum is 4.2V per cell. 
	& The measured voltage is not above 60 V.
	& 
	& \\ \hline
	AU2\_NF4
	& The capacity of the battery must not exceed 1,000 Wh.
	& The capacity of the battery is measured with a power analyzer.
	& The power analyzer shows a value below 1,000 Wh.
	& 
	& \\ \hline
	AU2\_NF5
	& The Motor Control Unit must be able to measure the vehicle's velocity with a precision of $\pm$2 km/h.
	& The vehicle is accelerated up to a series of constant speeds. The measurements, which are stored on an internal SD-card, are compared to the input.
	& The measured speed is identical to the vehicles speed.
	& 
	& \\ \hline			
\end{longtable}