\pagebreak
\section{Test of Non-functional requirements}

\begin{longtable}{|p{1.7 cm}|p{2.9 cm}|p{2.9 cm}|p{1.9 cm}|p{1.8 cm}|p{1.8 cm}|}
	\hline
	\textbf{Req. \#} & \textbf{Description} & \textbf{Test procedure} & 
	\textbf{Expected result} & \textbf{Actual result} & \textbf{Accept/ Comment} \\ \hline
	\endhead
	AU2\_NF1 
	& The horn must emit a sound equal or louder than 85 dBa when measured 4 meters horizontally from the vehicle.
	& The horn button is pressed and the sound pressure level is measured using a decibel-meter.
	& The decibel-meter shows a value above 85 dBa.
	& 
	& \\ \hline
	AU2\_NF2
	& The tone emitted by the horn must have a pitch between 420 Hz and 420 kHz.
	& The horn-button is pressed and the frequency is measured using an electronic tuner.
	& The measured tone has a frequency which lies between 420 Hz and 420 kHz.
	& 
	& \\ \hline
	AU2\_NF3
	& The voltage from the battery must not exceed 48 V nominal and 60 V maximum.
	& A multimeter is used to measure the maximum voltage of the battery. And note that the nominal voltage of the battery is 3.7V per cell, there is 12 cells in the battery, where the maximum is 4.2V per cell. 
	& The measured voltage is not above 60 V.
	& 
	& \\ \hline
	AU2\_NF4
	& The capacity of the battery must not exceed 1,000 Wh.
	& The capacity of the battery is measured with a power analyzer.
	& The power analyzer shows a value below 1,000 Wh.
	& 
	& \\ \hline
	AU2\_NF5
	& The Motor Control System must be able to measure the vehicle's velocity with a precision of $\pm$2 km/h.
	& The vehicle is accelerated up to a series of constant speeds. The measurements, which are stored on an internal SD-card, are compared to the input.
	& The measured speed is identical to the vehicles speed.
	& 
	& \\ \hline			
	AU2\_NF6
	& There must be backups for all PCBs used in AU2. And at least two sets of batteries and motors. So that if something breaks it is fast to install a working backup version.
	& Inspect the PCBs, their backup versions, the sets of batteries and motors.
	& There are two versions of each PCB, battery-pair and motor.
	& 
	& \\ \hline
	AU2\_NF7	& The battery must be placed on a metal tray in order to prevent an eventual battery-fire that would damage the system.
	& The car is inspected in order to determine if the battery is placed on a metal tray.
	& The battery is located on the metal tray.
	& 
	& \\ \hline
	AU2\_NF8 
	& The positive and negative circuits of the propulsion battery must be electrically isolated from the vehicle frame.
	& The car is inspected in order to determine if the systems are isolated from the frame.
	& The systems are isolated from the frame.
	& 
	& \\ \hline
	AU2\_NF9 
	& The battery must be installed outside the cockpit behind the bulk head.
	& The car is inspected in order to determine the location of the battery.
	& The battery is isolated from the driver's compartment and behind the bulk head. 
	& 
	& \\ \hline
\end{longtable}