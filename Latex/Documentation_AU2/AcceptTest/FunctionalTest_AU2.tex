\section{Test of functional requirements}

\begin{longtable}{|p{1.6 cm}|p{2.9 cm}|p{2.9 cm}|p{1.9 cm}|p{1.8 cm}|p{1.8 cm}|}
	\hline
	\textbf{Req. \#} & \textbf{Description} & \textbf{Test procedure} & 
	\textbf{Expected result} & \textbf{Actual result} & \textbf{Accept/ Comment} \\ \hline
	\endhead
	AU2\_F1 
	& The car must have one electric storage device, one motor and one control unit.
	& The car is inspected to determine whether or not it contains the specified parts.
	& The car contains one electric storage device, one motor and one control unit.
	& 
	& \\ \hline
	AU2\_F2 
	& The car must be equipped with a built-in horn mounted towards the front of the vehicle, which can be activated by the driver in the cockpit.
	& The car is inspected to determine if it contains the horn and the horn-button is pressed in order to check for functionality.
	& The car contains a horn which is activated when the button is pressed.
	& 
	& \\ \hline
	AU2\_F3 
	
	& The car must be equipped with an emergency shutdown mechanism, which must isolate the battery from the propulsion system when the button is pressed.
	& While the emergency shutdown button is pressed, then press the red button on the right hand steering handle.
	& Power is no longer supplied to any electrical system and the status-LED turns off.
	& 
	& \\ \hline
	AU2\_F4 
	& The car must include two "dead man's" safety switches which must be activated at all times in order for the car to drive.
	& Locate the safety switches that are mounted on the steering handle. Then hold the switches pressed while turning on the propulsion system, by pressing the red button on the right hand steering handle. Then release the safety switches. 
	& The car starts to move forward and then stops moving when the safety switches are released. 
	& 
	& \\ \hline
	AU2\_F5 
	& The positive and negative circuits of the propulsion battery  must be electrically isolated from the vehicle frame.
	& The car is inspected in order to determine if the systems are isolated from the frame.
	& The systems are isolated from the frame.
	& 
	& \\ \hline
	AU2\_F6 
	& The battery must be protected against electrical overload with a fuse.
	& Locate the fuse near the battery and confirm that it doesn't have a rating higher than 30A. 
	& The fuse is located near the battery and has a rating of 30A or less. 
	& 
	& \\ \hline
	AU2\_F7 
	& The battery must be installed outside the cockpit behind the bulk head.
	& The car is inspected in order to determine the location of the battery.
	& The battery is isolated from the driver's compartment and behind the bulk head. 
	& 
	& \\ \hline
	AU2\_F8 
	& The car must be equipped with connectors that fit a joulemeter, which is to be located between the battery and the motor controller. The display must be readable from outside the vehicle's body.
	& Inspect for joulemeter connectors inbetween the motor controller and and the battery and thus confirm that the meter is correctly installed. Then inspect the outside of the vehicle's body for the joulemeter's display.
	& The correct connectors are found between the battery and the motorcontroller and the display is located outside the vehicle's body.
	& 
	& \\ \hline
	AU2\_F9 
	& The car should be equipped with a data-log, which is able to collect data concerning the power from the battery and the wheel's angular velocity.
	& Locate a SD-card near the motor controller, unplug the SD-card and test on a computer if it has collected the required data.  
	& A SD-card is found near the motor controller and the required data is found on the SD-card.
	& 
	& \\ \hline
	AU2\_F10 
	& The car must be equipped with a speedometer.
	& Inspect the car's rear wheel and locate the speedometer.
	& The speedometer is found on the car's rear wheel. 
	& 
	& \\ \hline	
	AU2\_F11 
	& The car's BMS must be able to protect the cells of the battery from undervoltage and overvoltage. This requirement is already implemented in the BMS from earlier projects. (BMS\_F.1)
	& Connect a voltage between one of the BMS cell inputs and the reference pin. Try with both 1 V and 5 V. Test the two connections to "J9" on the analog front end board with a multimeter.
	& The multimeter shows less than 20 V.
	& 
	& \\ \hline
	AU2\_F12 
	& The car's BMS must be able to protect the battery from emitting overcurrent. This requirement is already implemented in the BMS from earlier projects. (BMS\_F.1)
	& Connect a powersource which is emitting 40 A. Test the two connections to "J9" on the analog front end board with a multimeter. (Replace the fuse with a new one after the test of this requirement.)
	& The multimeter shows less than 20 V.
	& 
	& \\ \hline
	AU2\_F13 
	& The car's BMS must be able to balance the cells if needed. This requirement is already implemented in the BMS from earlier projects. (BMS\_F.4)
	& Unbalance the cells of the battery. Connect the micro-USB to a PC and use for example Tera Term to see the cell voltages. Connect all the cells and wait for BMS to balance them again.
	& The cell voltages slowly belances out. 
	& 
	& \\ \hline
	AU2\_F14
	& The car's BMS must be able to protect the battery from overheating. The threshold when discharging is set to 60 degrees celsius maximum. And when charging, the threshold is set to 45 degrees celsius maximum.
	& Warm the batteries up to above 60 degrees celsius without connecting the charger. Test the two connections to "J9" on the analog front end board with a multimeter. Try also when charging the batteries.
	& The multimeter shows less than 20 V.
	& 
	& \\ \hline
	AU2\_F15
	& BMS should be able to transfer data to the motorcontroller via CAN-communication so the data can be collected on a SD-card.
	& Connect a CAN-reader and see if the results matches what it should. As specified in the earlier documentation of the BMS. \fxnote{reference to 2013BMS Documentation}
	& \fxnote{reference to 2013BMS Documentation}. 
	& 
	& \\ \hline	
\end{longtable}

%AU2\_F5 
%& The battery must be placed on a metal tray in order to prevent an eventual battery-fire that would damage the system.
%& Inspect the car to determine the location of the battery and confirm that it is placed on top of a metal tray. 
%& The battery is located on a metal tray. 
%& 
%& \\ \hline