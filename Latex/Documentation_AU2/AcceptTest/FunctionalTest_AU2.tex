\section{Test of functional requirements}

\begin{longtable}{|p{1.6 cm}|p{2.9 cm}|p{2.9 cm}|p{1.9 cm}|p{1.8 cm}|p{1.8 cm}|}
	\hline
	\textbf{Req. \#} & \textbf{Description} & \textbf{Test procedure} & 
	\textbf{Expected result} & \textbf{Actual result} & \textbf{Accept/ Comment} \\ \hline
	\endhead
	AU2\_F1 
	& AU2 must have one electric storage device, one motor and one control unit for the motor.
	& AU2 is inspected to determine whether or not it contains the specified parts.
	& AU2 contains one electric storage device, one motor and one control unit.
	& The body of AU2 is not completed. Therefore the test could not be executed successfully.
	& Failed.\\ \hline
	AU2\_F2 
	& AU2 must be equipped with a built-in horn mounted towards the front of the vehicle, which can be activated by the driver in the cockpit.
	& AU2 is inspected to determine if it contains the horn and the horn-button is pressed in order to check for functionality.
	& AU2 contains a horn which is activated when the button is pressed.
	& The body of AU2 is not completed. Therefore the test could not be executed successfully.
	& Failed.\\ \hline
	AU2\_F3 
	
	& AU2 must be equipped with an emergency shutdown mechanism, which must isolate the battery from the propulsion system when the button is pressed.
	& While the emergency shutdown button is pressed, then press the red button on the right hand steering handle.
	& Power is no longer supplied to any electrical system and the status-LED turns off.
	& The body of AU2 is not completed. Therefore the test could not be executed successfully.
	& Failed.\\ \hline
	AU2\_F4 
	& AU2 must include a "dead man's" safety switch which must be activated at all times in order for the car to drive.
	& Locate the safety switch that is mounted on the steering handle. Then hold the switch pressed while turning on the propulsion system, by pressing the green button on the left hand steering handle. Then release the safety switch. 
	& AU2 starts to move forward and then stops moving when the safety switch is released. 
	& The body of AU2 is not completed. Therefore the test could not be executed successfully.
	& Failed.\\ \hline
	AU2\_F5
	& AU2 must be equipped with spade connectors that fit a joulemeter, which is to be located between the Battery Management System(BMS) and the Motor Control System(MCS). The display must be readable from outside the vehicle's body.
	& Inspect for joulemeter connectors inbetween the MCS and the BMS and thus confirm that the joulemeter is correctly installed. Then inspect the outside of the vehicle's body for the joulemeter's display.
	& The correct connectors are found between the battery and the motorcontroller and the display is located outside the vehicle's body.
	& The connectors fit the joulemeter and are located between the BMS and MCS. But the body of AU2 is not completed. Therefore the test could not be fully executed.
	& Failed.\\ \hline
	& \\ \hline
	AU2\_F6 
	& AU2 should be equipped with a data-log, which is able to collect data concerning the power from the battery and the wheel's angular velocity.
	& Locate a SD-card near the MCS, unplug the SD-card and test on a computer if the SD-card is not empty.  
	& A SD-card is found near the MCS and data is found on the SD-card.
	& SD-card functions as expected. But the body of AU2 is not completed. Therefore the test could not be fully executed.
	& Failed.\\ \hline
	AU2\_F7
	& AU2 must be equipped with a speedometer.
	& Inspect the spokes on AU2's rear wheel for magnets.
	& Magnets are found on AU2's rear wheel. 
	& The body of AU2 is not completed. Therefore the test could not be executed successfully.
	& Failed.\\ \hline
	AU2\_F8
	& AU2's BMS must be able to protect the cells of the battery from undervoltage and overvoltage\cite{BMSDocumentation} (section 1.5.1.2).
	& Connect a voltage between one of the BMS cell inputs and the reference pin. Try with both 1 V and 5 V. Test the two connections to "J9" on the Analog Front End board with a multimeter.
	& The multimeter shows 0 V $\pm$0.5 V.
	& The multimeter shows 47.42 V.
	& Failed.\\ \hline
	AU2\_F9
	& AU2's BMS must be able to protect the battery from emitting overcurrent\cite{BMSDocumentation} (section 1.5.1.2).
	& Connect a powersource which is emitting 40 A and short circuit the speakon connector. Test the two connections to "J9" on the analog front end board with a multimeter. (Replace the fuse with a new one after the test of this requirement.)
	& The multimeter shows 0 V $\pm$0.5 V.
	& The multimeter shows 0.02 V.
	& Success.\\ \hline
	AU2\_F10
	& AU2's BMS must be able to balance the cells if needed\cite{BMSDocumentation} (section 1.5.1.2).
	& Unbalance the cells of the battery, by putting a load on just one cell until they are unbalanced. Connect the micro-USB to a PC and use for example Tera Term to see the cell voltages. Connect all the cells and wait for BMS to balance them again, by checking if the cell voltages becomes more and more similar.
	& The cell voltages slowly balance out. 
	& The cell voltages do not slowly balance out.
	& Failed.\\ \hline
	AU2\_F11
	& AU2's BMS must be able to protect the battery from overheating.\cite{BMSDocumentation} (section 1.5.2.2).
	& Warm the temperature sensor up to above 60 degrees celsius without connecting the charger. Test the two connections to "J9" on the Analog Front End board with a multimeter. Try also while charging the batteries.
	& The multimeter shows 0 V $\pm$0.5 V.
	& The multimeter shows 47.42 V.
	& Failed.\\ \hline
	AU2\_F12
	& AU2's BMS should be able to transfer data to the MCS via CAN-communication\cite{BMSDocumentation} (section 1.5.1.2).
	& Connect a CAN-reader and see if the results match the result in the previous documentation of the BMS2013\cite{BMSDocumentation} (section 4.2.2 page 91).
	& The CAN-reader should look like the figure\cite{BMSDocumentation} (section 4.2.2 page 91).
	& The CAN-reader did not look like the figure specified, because nothing was received.
	& Failed.\\ \hline	
\end{longtable}