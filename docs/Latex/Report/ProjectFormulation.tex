\chapter{Project formulation}
The purpose of this project is to create the electrical and software systems for the car dubbed \textbf{AU2}. The car is designed to compete in Shell Eco-Marathon (SEM) on behalf of AU and should be as energy-efficient as possible. In order to measure and optimize the car's energy-consumption, a dynamo-meter is to be developed in parallel with the car's electrical systems. 

The dynamo-meter is based on a previous system with similar functionality\cite{BAC_rullefelt} and some of the design can be re-used. The new dynamo-meter has been dubbed \textbf{Rolling Road} and is also part of the project.

In order to control and read the measurements done by Rolling Road, a third system must also being developed in parallel with the other two systems. This system is a GUI(Graphical User Interface) and is simply dubbed \textbf{Rolling Road GUI}. This is also part of the project.

In order to compete in SEM the car's electrical systems has to fulfil a number of requirements set down by Shell\cite{AU2}. Identifying and implementing the solutions to these are part of the project.

Both Rolling Road and AU2 have been designed using former documentations\cite{BAC_rullefelt}\cite{BAC_zenith33}. These documentations are made by both electrical- and mechanical-engineers. Identifying and using the design-parameters from these documentations is also a part of this project.

\newpage
\section{SEM Parameters }
This project differs from previous projects on Aarhus School of Engineering as the team has to work together with a team of mechanical engineering students in order the create AU2. Collaboration with the mechanical engineering students is vital for a successful project. The project should culminate in the both teams participation in SEM 2016 with AU2. In order to meet the qualifications made by Shell and Aarhus School of Engineering, the team also has to complete the following points:
\begin{itemize}
	\item Complete the electrical systems in AU2 in order to guarantee full functionality during Shell Eco-Marathon and installing the systems in the car.
	\item Having functional backup-circuits of every electrical system in the car.
	\item Creating a technical documentation which met the requirements set by Shell.
	\item Optimizing the car's driving-algorithm in order make the car as efficient as possible.
	\item Equipping the pit area with the necessary tools in order to maintain the car on site, during Shell Eco-Marathon.
	\item Organizing the trip to London, lodging of the participants during Shell Eco-Marathon and other necessary points.
	\item Organizing the team's duties on site during Shell Eco-Marathon.
\end{itemize}
Most of these points are not vital to the project and has deadlines after project turn-in on the 27\textsuperscript{th} of May 2016. The points should nonetheless still be considered problems which must be kept in the project's backlog.