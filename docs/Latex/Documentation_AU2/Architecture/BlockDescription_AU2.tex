\section{Block description}
The blocks in the BDD are described as follows:

\begin{itemize}
	\item \textbf{Rolling Road}\\
	Description: A secondary system which has been designed to test the efficiency of the driving algorithm.
	\item \textbf{Joulemeter}\\
	Description: A joulementer which measures the amount of energy consumed by the Propulsion Motor. This block is only part of the system during the SEM.
	\item \textbf{Battery Holder}\\
	Description: A battery holder for the two batteries, which will be implemented in a serial connection and thereby understood as one battery. The battery implemented is a 2 times 6 celled battery with a nominal voltage on 3.7 V per cell. The battery is used as the power source.
	\item \textbf{Battery Management System}\\
	Description: The system manages the Propulsion Battery whilst protecting the circuits and the battery from shorting. This is done by constantly surveying the current and temperature from the battery. If any parameter exceeds a specified threshold, the system will isolate the battery.
	\begin{itemize}
		\item \textbf{Current Sensor}\\
		Description: Senses value and direction of battery current, and presents this information to the Analog Front End.
		\item \textbf{Analog Front End}\\
		Description: This unit handles analog to digital conversions from the batteries, the temperature sensor as well as the current sensor.
		\item \textbf{Digital Unit}\\
		Description: The Digital Unit commands the Analog Front End on what calculations to execute and when to collect data. Furthermore, it can communicate with the Motor Control System by using the CAN protocol and it has a built in USB interface, which can be used to debug on. 
		\item \textbf{DC/DC Converter}\\
		Description: The DC/DC Converter is responsible for converting a high voltage power supply to a lower voltage power supply.
		\item \textbf{Isolation Switch}\\
		Description: This unit is a relay that is controlled by the Battery Management System and the Emergency System. It has the ability to stop the propulsion of the vehicle.
		\item \textbf{Analog Front End Extension Module}\\
		Description: This unit allows the Analog Front End to measure more than 6 cells. 
		
	\end{itemize}
	\item \textbf{Horn}\\
	Description: A sound-making device which emits a tone when the driver activates it.
	\item \textbf{Distribution Block}\\ 
	Description: Converts the 48 V from the battery to recommended levels in order to power various systems in the car.
	\item \textbf{Motor Control System}\\
	Description: The system controls the driving-algorithm which the car operates by. Furthermore, it logs various parameters about the car's current run (velocity and power-consumption) on to a removable SD-card.
	\begin{itemize}
		\item \textbf{CAN Tranceiver}\\
		Description: Allows connection between the Motor Controller and a computer using the CAN-protocol.
		\item \textbf{Speedometer}\\
		Description: Constantly measures the car's current speed in order to let the car follow the implemented driving-algorithm.
		\item \textbf{SD-Logger}\\
		Description: Logs various parameters about the current run to a removable SD-card which may be examined to optimize the driving algorithm.
		\item \textbf{Power Control}\\
		Description: Controls the amount of current delivered to the motor.
		\item \textbf{Motor Controller}\\
		Description: Controls the power consumption of the Propulsion Motor in order to let the car follow the implemented driving-algorithm. 
	\end{itemize}
	\item \textbf{Propulsion Motor}\\
	Description: The electrical motor which propels the car forwards.
	\item \textbf{Emergency System}\\
	Description: A system  consisting of two mechanical switches  which must be activated, in order for the system to be turned on.
	\item \textbf{Computer}\\
	Description: An optional block which can be connected in order to measure various parameters or calibrate the system.
\end{itemize}