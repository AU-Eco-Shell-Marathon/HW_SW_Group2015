\input{../Preamble}
\newcommand{\HRule}{\rule{\linewidth}{0.1mm}}

\begin{document}
	\begin{center}
		{\huge \bfseries \textsc{Møde referat nr. XX}}\\
		\textsc{\large 4. semesterprojekt - Gruppe 1}\\[0.3cm]
	\end{center}
	\begin{tabular}{ll}
	\large \textbf{Dato:} & 24/08/2015  \\ % Sted
	\large \textbf{Tid:}  & 10:15-11:00 \\ % Tid
	\large \textbf{Sted:} & Nygaard		\\ % Lokation
	\large \textbf{Deltagere:} & Jens (JN), Jonas (JH), Jonathan (JS), Laimonas (LB), \\
	\large \textbf & Thomas (TN),  Thomas (TS) \& Carl Jakobsen (JA)\\
	\large \textbf{Udeblevet:} & Navn her (Syg) \& Navn her (Ingen udmelding)	\\
	\end{tabular}\\
	\phantom{\,}\hspace{0.1em} \large \textbf{Referat:}
	\begin{enumerate}
		\itemsep 0.3em 
		\item Mødeleder\\
			Thomas Nielsen
		\item Referent\\
			Thomas Rasmussen
		\item Godkendelse af sidste referat\\
			Dette er det første referat.
		\item Opfølgning på aktionspunkter
		\begin{itemize}
			\itemsep 0.3em 
			\item Tur til Frankrig\\
				Der er ikke den store tiltag til en Frankrig-tur fra E'ernes side. Det er maskin-Anders' der står for turen.\\
				Skolen har ikke noget at gøre med turen.
			\item Møde med maskin-ingeniøre\\
				Der er problemer idet maskin-Anders har stået for kommunikationen mellem M- og E-ingeniøre.\\
				Carl arrangerer mødet med M-ingeniørene. Han regner med, at det bliver tisdag.
			\item Den tidligere elbil står måske i Herning\\
				Carl mener, at bilen vil blive leveret tilbage til Navitas inden for de nærmeste par dage.
			\item Der er yderligere 2 elektro-folk der bygger på bilen\\
				De andre E-ingeniøre arbejder med en børsteløs DC-motor som måske er mere effektiv end den nuværende motor. Hvis dette er tilfældet skal motoren i bilen skiftes. Dette vil dog ikke have betydning for vores projekt.
			\item Fremtidige vejledermøder
				Carl vil se på hvilke dage der passer ham bedst. Det vil nok under alle omstændigheder blive omkring kl. 16:00.
			\item I-eksamen for Jonas\\
				Carl overvejer hvordan Jonas' eksamen bør behandles i forhold til, at der er tale om et E-projekt. Han regner dog ikke med, at der bliver nogle problemer.
			\item Batterierne er ødelagte\\
				Der kan købes nye batterier. Minimum 3 par.
			\item Carl lægger den tidligere dokumentation op
			\item Hvordan kommer vi videre?\\
				Enheder fra tidligere projekter skal testes og der skal bedømmes hvad der skal ombygges og hvad der kan godkendes.\\
				Få lavet et overblik over hvilke problemer der skal løses for, at få lavet en ordentlig motorstyring, rullefelt og batteri-management-system.\\
				Vi bør kraftigt overveje et bærbart værksted med reservedele som kan medbringes til banen i London. Hvilke komponenter bør medbringes.
			\item Hvilket µ-controller skal bruges?\\
				Dette er et valg vi selv tager.
			\item Hvilke ting må forbindes til hvilke batterier?\\
				Der kan sendes en mail til Shell omkring hvad der må forbindes til hvilke batterier.
			\item Kageordning\\
				Det blev et nej.
			\item SCRUM\\
				Der holdes SCRUM-møder hver mandag (kl. 12:00 i Edison 140) og fredag (kl. 11 i Shannon ved stolene). 
			\item ZEN-hub\\
				Der bliver brugt ZEN-hub til projektstyring. Jonas står for opsætning af det.
		\end{itemize}
		\item Update fra grupperne\\
			Der er ingen grupper endnu.
		\item Tidsplan\\
			Der er ikke lavet en tidsplan endnu.
		\item Nye aktionspunkter til næste møde
		\begin{itemize}
			\itemsep 0.3em 
			\item Lav en nogenlunde arbejdsfordeling.
			\item Tjek ZEN-hub for opgaver.
		\end{itemize}
		\item Tidspunkt for næste møde\\
			Tirsdag d. 9/2 kl. 14:15-15:15
		\item Kagebager til næste møde\\
			Ikke mig
		\item Evt.\\
			Yeah right.
	\end{enumerate}
\end{document}