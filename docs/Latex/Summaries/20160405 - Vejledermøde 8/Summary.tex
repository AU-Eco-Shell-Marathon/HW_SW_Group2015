\input{../Preamble}
\newcommand{\HRule}{\rule{\linewidth}{0.1mm}}

\begin{document}
	\begin{center}
		{\huge \bfseries \textsc{Møde referat nr. 08}}\\
		\textsc{\large 4. semesterprojekt - Gruppe 1}\\[0.3cm]
	\end{center}
	\begin{tabular}{ll}
	\large \textbf{Dato:} & 05/04/2016  \\ % Sted
	\large \textbf{Tid:}  & 14.15-16.00 \\ % Tid
	\large \textbf{Sted:} & Kahn bygningen	\\ % Lokation
	\large \textbf{Deltagere:} & Jens (JN), Jonas (JH), Jonathan (JS), Laimonas (LB), \\
	\large \textbf & Thomas (TN),  Thomas (TS) \& Carl Jakobsen (JA)\\
	\large \textbf{Udeblevet:} & Navn her (Syg) \& Navn her (Ingen udmelding)	\\
	\end{tabular}\\
	\phantom{\,}\hspace{0.1em} \large \textbf{Referat:}
	\begin{enumerate}
		\itemsep 0.3em 
		\item Mødeleder\\
			Jens-Peter
		\item Referent\\
			Laimonas
		\item Godkendelse af sidste referat\\
			Møde referat 07 gennemgås ved næste møde: 12 april. 

		\item Opfølgning på aktionspunkter
		\begin{itemize}
			\itemsep 0.3em 
			\item Ingen
		\end{itemize}
		
		\item Generelle ting
			\begin{itemize}
				\item Rullefelt demonstration (E-ingeniører)
			
				\begin{itemize}
					\item Finde modstand i transmissionen – Køre den op i hastighed også slå el-motoren fra, så kan man se på kurven over nedløbs tid og derfra aflæse momentetet under nedløbet => fornemmelse for  rulningsmodstanden, og måske tomgangsmodstanden. 
					\item Der er mindre modstand når hjulet kører på et plan, i modsætning  til på en rulle.
				\end{itemize}
				\item Stand til bilen / teste rullefeltet (M-ingeniørerne)
			
				\begin{itemize}
					\item Snak om evt. metal bord som skal indk\o{}bes/fremstilles og dertil laves huller
					i, s\aa{} rullefeltet kan fastmonteres og bilen testes p\aa{} den m\aa{}de, mens
					man samtidig kan arbejde p\aa{} bilen.
					
					\begin{itemize}
						\item Need to have, good to have m.m.
						\item Stort set M-erne der har ansvar for dette -$>$ Der foretages en brainstorm og lign.
					\end{itemize}
				\end{itemize}
			\end{itemize}
			
			\begin{itemize}
				\item Status på kobling og bære-plade til direkte test af DC-motorerne, (M-ingeniører)
					\begin{itemize}
						\item Der vil tilføjes en kobling, som kan få lov til at sidde der permanent, og være mulighed for at skifte motoren ud og ind.
					\end{itemize}
			\end{itemize}
				
			\begin{itemize}
				\item Status på øvrige E til DC-motoren, (E-ingeniører)
					\begin{itemize}
						\item DC-motoren er kommet hjem, de siger, at de kan styre den – Den burde være klar, når pladen er klar.
						\item BMS(Battery Management System) forventes at være klar, og fungerende om nogle ugers tid.
						\item MCU – Koden er klar, der mangler nogle tests, og der tænkes at ses på det.
					\end{itemize}
			\end{itemize}
			
			\begin{itemize}
				\item Status på monocoque (M-ingeniører)
					\begin{itemize}
						\item Grundet uheldige instanser kommer der en forsinkelsen, fordi overfladen skal glattes ud, da den er meget grov.
						\item Man satser på at være effektive som om man skulle afsted til Frankrig, men man når højst sandsynligt ikke at få bilen til at køre et test løb før til London. 
					\end{itemize}
			\end{itemize}
	
			\begin{itemize}
				\item Status på rutens belastningsvariationer, (M-ingeniører)
					\begin{itemize}
						\item Er blevet udarbejdet, og findes
						\href{<https://www.dropbox.com/s/rn7g23q643xp2ua/Topographie\%20of\%20the\%20track\%20WITH\%20Force\%20Torque\%20and\%20Power.xlsx?dl=0>}{HER}
					\end{itemize}
			\end{itemize}
			
			\begin{itemize}
				\item Planlægning af rejsedage Frankrig, fælles
				\begin{itemize}
					\item Der tages ikke afsted til Frankrig.
				\end{itemize}
			\end{itemize}		
			
			\begin{itemize}
				\item Planlægning af rejsedage London, fælles
					\begin{itemize}
						\item I løbet af ugen vil London teamet sammensættes.
							\begin{itemize}
								\item Man skriver sig ind i et Excel dokument inden fredag aften (8 april). (David)
							\end{itemize}
						\item Da man ikke skal af sted til Frankrig, foreligger der et større budget til rejseudgifter. 
						\item Der er blevet talt om at sove i hotel, som ligger på ca. 250kr pr. mand pr. dag, hvilket vil være omkring 21000kr for 12 personer for 1 uge. (ingen ansvarlige)
						\begin{itemize}
							\item Der ligger nogle hoteller i gå afstand fra stedet hvor vi  skal være, som er okay i pris.
						\end{itemize}
						\item Der vil derefter også bestilles fly biletter, det koster ca. 500kr tur/retur fra mandag-mandag(27juni-4juli). (Ingen ansvarlige)
						\item Det gælder om at bestille hurtigts muligt, da der er mange som deltager samt det er billigst nu, end senere. 
					\end{itemize}
			\end{itemize}
			
			\begin{itemize}
				\item Eventuelt
				\begin{itemize}
					\item David har fået job fra d. 18 april i Kolding
						\begin{itemize}
							\item Hvis det kan holdes i weekender, vil han kunne være der.
							\item Hverdage / eftermiddage ikke muligt fremmøde -> konsulent.
						\end{itemize}
					\item Tøj bestilling / aftale med firmaet (David)
					\item Næste møde aftale efter behov. 
				\end{itemize}
			\end{itemize}
						
		\item Tidsplan\\
			Sprint 5 er igangværende. 

		\item Tidspunkt for næste møde\\
			12/04/2016 - 14.15-15.15
	\end{enumerate}
\end{document}