\documentclass[10pt,a4paper]{article}
\usepackage[latin1]{inputenc}
\usepackage{amsmath}
\usepackage{amsfonts}
\usepackage{amssymb}
\usepackage{graphicx}
\author{Jonathan}
\title{Applikations model - Rolling Goal - M�le enhed}
\begin{document}
	\section{Applikationsmodel Rolling Goal - m�leenhed.}

	Rolling Goal m�leenhed er den enhed der fortager alle m�linger og regulering p� Rolling Goal, ud fra de paramter som der bliver sendt fra kontrol enheden. M�leenheden her en r�kke forskellige sensor, der overv�ge str�mmen, sp�ndingen, Moment og hastighed.
	
	\subsection{Klassediagram} 


\begin{figure}[h]
\centering
\includegraphics[width=0.7\linewidth]{"applikationsmodel"}
\caption[Klassediagram for Rolling Goal]{Klassediagram for Rolling Goal.}
\label{fig:applikationsmodel}
\end{figure}

	P� figur \ref{fig:applikationsmodel} kan klassediagrammet ses. Der er en Kontrollerklasse og 3 boundary klasser.\\
	Sensor klassen bruges til at opsamle data fra sensorerne, og klarg�re dataet til b�de PID styringen og til kontrol enheden igennem UART'en.\\
	PID klassen bruges til at regulere belastningen p� generatoren, s� det �nsket moment bliver fastholdt.\\
	UART klassen bruges til kommunikation mellem m�leenheden og kontroller enheden. 
	
	
\end{document}